\chapter{Problemstellung}\label{ch:data}

\section{Analyse}

\subsection{Besonderheiten beim IT-Dienstleister kubus IT}
Die kubus IT ist durch ihre Funktion als Dienstleister für die gesetzlichen Krankenkassen AOK Bayern und AOK PLUS teil der öffentlichen Verwaltung.

Durch diese Position ergeben sich rechtliche Besonderheiten im Vergleich zu einem vergleichbaren Dienstleister der deutschen Wirtschaft.

[...]

\subsection{Vergabeverfahren im für Vendoren}
Der Abteilung Einkauf ist die Abteilung Vendor-Management zur Orchestrierung der Beziehungen zwischen der kubus IT und Anbietern.

Die Anbieter werden in drei Kategorien segmentiert.

\begin{itemize}
\item[-] A-Vendoren: Große Abhängigkeit, hohe Kritkalität und fehlende kurzfristige Austauschbarkeit
\item[-] B-Vendoren: Mittlere Vorlaufeiten und Kosten für den Austausch
\item[-] C-Vendoren: Anbieter für Standardleistungen mit leichter Austauschbarkeit
\end{itemize}

Diese Anbieterkategorien sind nicht ausschließlich für Anbieter von Cloud-Computing konzepiert. Stattdessen wird bei jeder Geschäftsbeziehung im Vendoren-Management mit diesem Schema gearbeitet.

Demnach zählen folgende Anbieter aktuell in das A-Segment:

\begin{itemize}
\item[-] Arvato: Cloud-Computing
\item[-] DATAGROUP: "IMAC"
\item[-] Avaya: Cloud-Kommunikation
\end{itemize}

\subsection{Bisherige Wechsel von Anbietern}


