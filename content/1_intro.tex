\chapter{Einleitung}\label{ch:intro}

\section{Abgrenzungen}

\subsection{Definition von technischen Kriterien}
Vendor-Lockin ist ein primär technisches Problem für den Käufer eines Produktes beziehungsweise konkret eins Cloud-Computing-Anbieters.

Daher wird untersucht welche technischen Kriterien zu diesem technischen Problem führen.

Technische Kriterien sind Eigenschaften eines Produktes im Kontext von Cloud-Computing, die sich auf die inherente Struktur und die Bestandteile des Produktes beziehen.
Außerdem sind Schnittstellen des Produktes zu anderen Produkten gemeint.

\subsection{Distanzierung von ökonomischen Ansätzen}
Im Gegensatz dazu sind vertragliche oder ökonomische Kriterien Gegenstand dieser Arbeit.

Zur Verdeutlichung wird also beispielsweise nicht untersucht, ob die These, dass das Nutzen eines teurerer Cloud-Computing-Anbieter seltener zum Vendor-Lockin führt, zutrifft.


\subsection{Auswahl der Cloud-Computing-Anbietern}
Bei der Auswahl der Anbieter wurden sowohl solche berücksichtigt, die die kubus IT bereits verwendete, als auch solche die vermeintlich interessante Ergebnisse liefern sollten.

Aktuell sind folgende Cloud-Computing-Anbieter bereits in Benutzung.
\listadd{Test}{Microsoft Azure}
\listadd{Test}{Arvato}
\listadd{Test}{Test}

Darüberhinaus werden folgende Anbieter aufgrund ihrer Relevanz auf dem internationalen Markt, ihrer besonderen Größe oder ihrer Relevanz für deutsche Firmen berücksichtigt. 
\listadd{Test}{Amazon Warehoue Services (Vereinigte Staaten)}
\listadd{Test}{Google Cloud Plattform (Vereinigte Staaten)}
\listadd{Test}{Alibaba Cloud (China)}
\listadd{Test}{IONOS Cloud (Deutschland)}


\section{Motiviation}

\subsection{Einordnung der Thematik in das aktuelle Geschehen}
Grundlegend sind die Argumente für das Cloud-Computing stärker geworden.
Der Grund hierfür sind die veränderten Anfordungen an IT-Infrastruktur.
Diese Infrastruktur muss immer widerstandsfähiger gegen Angriffe sein.
Außerdem muss diese leistungsfähiger sein, um die steigenden Datenmengen und deren Verabeitung durch künstliche Intelligenz zu ermöglichen.
Zuletzt steigt die Dezentralisierung der Arbeitswelt mit mobiler Arbeit und international zerstreuten Unternehmen.

Durch diese Gegebenheiten benötigt die Infrastruktur jeder Firma größer angelegte Sicherheitsmechnanismen, höhere Kapizitäten, mehr Flexibilität und Standortunabhänigkeit.

Um diese Anforderungen mit On-Premise-Lösungen zu erfüllen, sind erhebliche initale und fortlaufende Invetitionen notwendig.

Um diese Kosten zu vermeiden, die Risiken zu vermindern und um die Energie anderweitig zu fokussieren, setzten Unternehmen generell auf Cloud-Computing-Anbieter.
Diese Nutzen Spezialisierung und Skalierung, um die oben genannten Anforderungen zu erfüllen.

Durch allgemeine Lösungen können Kunden dieser Anbieter von niedrigeren Kosten im Vergleich zu eigenen Lösungen profitieren.

Zu beachten ist jedoch, dass gegebenenfalls vergleichsweise höhere Kosten für Anpassungen an den allgmeinen Lösungen anfallen.

Dazu kommt, dass nicht vollständig ohne das Personal, das bei einer On-Premise-Lösung notwendig ist, kalkuliert werden sollte, da sonst wichtige Berater auf Seite der eigenen Firma und Fachwissen bei Umstiegen, Wartungen, Anpassungen oder Integration von neuen Bausteinen auf Firmenseite verloren gehen kann.
Darüberhinaus entstehen so Abhängigkeiten zu den Cloud-Computing-Anbietern entsteht.

Besonders hervorzuheben ist der letzte Punkt, denn die Schwierigkeit von Umstiegen auf andere Anbieter wird von Cloud-Computing-Kunden unterschätzt.

Ein Umstieg auf einen alternativen Anbieter, kann notwendig werden, wenn die Preise des aktuellen Anbieters nicht mehr tragbar sind oder andere Anbieter ein besseres Preis-Leistungs-Verhältnis anbieten.

Nicht nur finanzielle Aspekte können zu einem Wechselwunsch beim Kunden führen.

Durch Anpassungen am Leistungskatalog und die vertragliche Möglichkeit manche Leistungen nicht mehr anzubieten, kann es dazu kommen, dass notwenige Bausteine nicht mehr vom Cloud-Computing-Anbieter unterstützt werden.

Außerdem kann es dazu kommen, dass Kunden von mehreren Anbietern ihre benötigten Leistungen auf einen einzigen konsolidieren wollen oder im Gegenbeispiel ihre Anforderungen auf mehrere Anbieter verteilen wollen, um die unterschiedlichen Alleinstellungsmerkmale mehrerer Anbieter gleichzeitig zu nutzen.

Zuletzt kann es auch durch äußere Faktoren wie gesetzliche Vorgaben, denen das Produkt des aktuellen Anbieters nicht mehr folgt, dazu kommen, dass ein Wechsel unbedingt notwendig wird.
Auch geopolitische Änderungen wie Zölle zählen zu den Gründen für einen Wechselwunsch.

\subsection{Risiken bei Anbieterbindung}
Existiert bei einem Cloud-Computing-Kunden nun der Wechselwunsch, so müssen alternative Anbieter auf ihre Brauchbarkeit untersucht werden. 
\blindtext

