\chapter{Ausblick}\label{ch:ausblick}

\section{Handlungsempfehlung}

\subsection{Implementierung in den Vergabeprozess}
[…]

\section{Reaktion des Gesetzgebers}
\subsection{Gesetzliche Lösungen für Anbieterbindung}
Das Bewertungsmodell ist eine Lösung, die bei der Suche eines neuen Anbieters in einzelen Unternehmen wie der kubus IT eingesetzt werden kann.
\paragraph{Schutz der Kunden vor Anbieterbindung}
Aus Sicht eines Cloud-Anbieters ist die Bindung seiner Kunden natürlich erstrebenswert. 
Und hat ein Anbieter im Vergleich zu den anderen Optionen auf dem Markt besonders attraktive und einzigartige Produkte, sodass der Kunde gar nicht wechseln kann, weil diesem sonst die Produkte fehlen, dann profitiert im gewissen Sinne auch der Kunde.
Das ist jedoch nicht in erster Linie der Kern von Anbieterbindung.
Vielmehr wird durch gezielte Ausgestaltung der Produkte eine Abhängigkeit aufgebaut, sobald der Anbieter einmal gewählt ist, was dem Kunden nur schadet, da es diesen einschränkt.
So könnte man ein Gesetz gegen Anbieterbindung rechtfertigen.
\paragraph{Gesetzliche Möglichkeiten}
Anbieterbindung gesetzlich zu vollständig verhindern ist schwierig, da eine viele Faktoren zu einer Anbieterbindung beitragen. 
Daher liegt es nahe konkret einzelne Faktoren zu minimieren.
Das ginge beispielsweise durch eine Pflicht zur Interoperabilität. So müssten zwangsläufig Daten in universellen Formaten gespeichert werden, damit diese leichter in Anwendungen neuer Anbieter importiert werden können.

\paragraph{Aktuelle gesetzliche Lösung}
In der europäischen Union gilt seit dem 12. September 2025 der sogenannte Data Act. Dieser entält unter Anderem "Regeln und entsprechende Interoperabilitätsvorgaben, die es Kunden von Datenverarbeitungsdienste erleichtern, zu einem anderen Datenverarbeitungsdiensten zu wechseln oder verschiedene Datenverarbeitungsdienste parallel zu nutzen. Bei Datenverarbeitungsdiensten handelt es sich vorallem um Cloud-Dienste. \parencite{DataActBNA2026}
Der Data Act regelt folgende \glqq Mindestanforderungen an Verträge\grqq :
\begin{itemize}
\item[-] Wechselrecht
\item[-] Unterrichtungs- und Unterstützungspflichten
\item[-] Kündigungsfristen und Kompensation bei vorzeitiger Kündigung
\item[-] Konkretisierung des Wechselprozesses
\end{itemize}
\parencite{DataActCross2026}
Darüber hinaus muss ein Anbieter auf seiner Website offenlegen welche Maßnahmen getroffen wurden, um den Zugriff auf in der EU gespeicherte nicht-personenbezogene Daten durch andere Staaten zu verhindern. 
\parencite{DataAct2025}
Zuletzt werden Gebühren für die Unterstützung beim Wechsel des Cloud-Anbieters durch Artikel 29 des Data Act schrittweise verboten.
\parencite{DataAct2025}

\paragraph{Einordnung des Data Act}
Zum Zeitpunkt der Themenfindung für diese Arbeit, stand das Inkrafttreten des Data Act unmittelbar bevor. Trotzdem war eine gesetzliche Lösung für die Anbieterbindung zum Zeitpunkt der Verfassung zunächst unbekannt.

Der Data Act schränkt die Vertragsfreiheit für den Anbieter ein, was im Artikel von Samuel Cross krisiert wird. Zudem  \parencite{DataActCross2026}




\subsection{Effekt der aktuellen Gesetzesänderungen}

\subsection{tbd}
[…]