\chapter{Ausblick}\label{ch:ausblick}

\section{Handlungsempfehlung}
\subsection{Integration des Prozesses}
Auf Basis der Ergebnisse der Analyse wird empfohlen die Untersuchungen von Anbietern anhand des demonstrierten Prozesses zu etablieren.
Da die praktische Auswertung beim Arbeiten mit dem jeweiligen Produktkataloges des Cloud-Computing-Anbieters möglich ist,
bietet es sich an die Auswertung beiläufig bei der Ermittlung von Kosten und Produkt-Umfang zu ermitteln.
\subsection{Evaluierung durch praktische Erkenntnisse}
Nach der Integration des Prozesses, können bei künftigen Migrationsprozessen in der kubus IT
konkrete Kosten und Aufwände zu den jeweiligen Bewertungen ermittelt werden.
Die Evaluierung des Bewertungsmodells in der Praxis wird nach dieser Arbeit weiter verfolgt.

\section{Alternative Herangehensweise}
Zum Abschluss dieser Arbeit wird ein Lösungsansatz aus Perspektive des Gesetzgebers im Gegensatz zum Unternehmer betrachtet und kurz eingeordnet.
\subsection{Gesetzliche Lösungen für Anbieterbindung}
Das Bewertungsmodell ist eine Lösung, die bei der Suche eines neuen Anbieters in einzelne Unternehmen wie der kubus IT, eingesetzt werden kann.
\paragraph{Schutz der Kunden vor Anbieterbindung}
Aus Sicht eines Cloud-Anbieters ist die Bindung seiner Kunden natürlich erstrebenswert. 
Und hat ein Anbieter im Vergleich zu den anderen Optionen auf dem Markt besonders attraktive und einzigartige Produkte, sodass der Kunde gar nicht wechseln kann, weil diesem sonst die Produkte fehlen, dann profitiert im gewissen Sinne auch der Kunde.
Das ist jedoch nicht in erster Linie der Kern von Anbieterbindung.
Vielmehr wird durch gezielte Ausgestaltung der Produkte eine Abhängigkeit aufgebaut, sobald der Anbieter einmal gewählt ist, was dem Kunden nur schadet, da es diesen einschränkt.
So könnte man ein Gesetz gegen Anbieterbindung rechtfertigen.
\paragraph{Gesetzliche Möglichkeiten}
Anbieterbindung gesetzlich zu vollständig verhindern ist schwierig, da eine viele Faktoren zu einer Anbieterbindung beitragen. 
Daher liegt es nahe konkret einzelne Faktoren zu minimieren.
Das ginge beispielsweise durch eine Pflicht zur Interoperabilität. So müssten zwangsläufig Daten in universellen Formaten gespeichert werden, damit diese leichter in Anwendungen neuer Anbieter importiert werden können.

\paragraph{Aktuelle gesetzliche Lösung}
In der europäischen Union gilt seit dem 12. September 2025 der sogenannte Data Act. Dieser enthält unter Anderem Regeln und entsprechende Interoperabilitätsvorgaben, die es Kunden von Datenverarbeitungsdienste erleichtern, zu einem anderen Datenverarbeitungsdiensten zu wechseln oder verschiedene Datenverarbeitungsdienste parallel zu nutzen. Bei Datenverarbeitungsdiensten handelt es sich vor allem um Cloud-Dienste. \parencite{DataActBNA2026}
Der Data Act regelt folgende \glqq Mindestanforderungen an Verträge\grqq :
\begin{itemize}
\item[-] Wechselrecht
\item[-] Unterrichtungs- und Unterstützungspflichten
\item[-] Kündigungsfristen und Kompensation bei vorzeitiger Kündigung
\item[-] Konkretisierung des Wechselprozesses
\end{itemize}
\parencite{DataActCross2026}
Darüber hinaus muss ein Anbieter auf seiner Website offenlegen welche Maßnahmen getroffen wurden, um den Zugriff auf in der EU gespeicherte nicht-personenbezogene Daten durch andere Staaten zu verhindern. 
\parencite{DataAct2025}
Zuletzt werden Gebühren für die Unterstützung beim Wechsel des Cloud-Anbieters durch Artikel 29 des Data Act schrittweise verboten.
\parencite{DataAct2025}

\chapter{Ergebnisse}\label{apendix-scoringmodel}
Auf den nächsten Seiten folgen die konkreten Ergebnisse des Bewertungsmodells.

