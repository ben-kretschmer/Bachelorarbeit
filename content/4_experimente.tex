\chapter{Experimente}\label{ch:experimente}
In diesem Kapitel der Arbeit wird der Lösungsansatz gestestet, indem Cloud-Computing-Anbieter getestet und die Ergebnisse anschließend eingeordnet werden.
\label{krit-kat-experimente}
\section{Modell-Prüfung}
Die detailierten Ergebnisse der Prüfung mit dem Formular, das in Microsoft-Excel erstellt wurde, sind im Anhang der Arbeit aufgeführt.
\subsection{Einheitliche Gewichtung}
Im Kontext dieser Arbeit wird eine gleiche Gewichtung aller Kriterien im Katalog konfiguriert.
\begin{figure}
\begin{center}
\includegraphics[width=0.75\linewidth]{figures/Gegen-der-VLI-Bewertungen}
\caption{Resultate im Hinblick auf die Gesamtbewertung}
\label{vendor-lock-in-score}
\end{center}
\end{figure}
\paragraph{Beschreibung der Bewertungen}
Die Abbildung \ref{vendor-lock-in-score} zeigt fünf Säulen in einem Säulendiagramm mit der Überschrift \glq Gegenüberstellung der Vendor-Lock-In-Bewertungen\grq .
An der x-Achse sind die untersuchten Anbieter aufgelistet,
 wobei es sich bei Google Plattform, AWS und Azure um die führenden Anbieter im Cloud-Computing-Bereich handelt.
Die übrigen Anbieter sind IBM und Alibaba, die deutlich kleinere, aber dennoch signifikante Anteile des restlichen Marktes besitzen. 
(vergleiche \ref{lead-cc-A})
An die y-Achse markiert den erreichten Vendor-Lock-In-Score, wobei in dieser Form ein niedriger Score (Bewertung) für ein geringeres Risiko zur Anbieterbindung steht.
Die Anbieter bewegen sich alle im Bereich zwischen 60\% und 80\%, 
wobei die größte Differenz zwischen \glq IBM's Cloud\grq\ und \glq Microsoft's Azure\grq herrscht. 
In diesem Vergleich ist die Bewertung von Microsoft 15\% besser als die Bewertung von IBM.
\paragraph{Beschreibung weiterer Ergebnisse}
Die Abbildung \ref{oss-score} zeigt, wie auch die Gegenüberstellung der Gesamtbewertungen, fünf Säulen und die fünf verglichenenen Anbieter.
In dieser Abbildung ist jedoch auf der y-Achse der Anteil der Anwendungen aufgezeichnet, 
die auf der Basis von Open-Source-Technologie gebaut wurden oder selbst als Open-Source-Anwendung vom Cloud-Computing-Anbieter angeboten werden.
Die Anbieter liegen im Bereich zwischen 4\% und 15\%, wobei \glq Google’s Cloud\grq\ 11\% weniger Open-Source-Anwendungen aufweißt als \grq Microsoft's Azure\grq .

\begin{figure}
\begin{center}
\includegraphics[width=0.75\linewidth]{figures/Gegen-der-OSS-Scores}
\label{oss-score}
\caption{Ergänzende Ergebnisse im Hinblick auf Open-Source-Nutzung bei den Produkten}
\end{center}
\end{figure}

\section{Interpretation}
\paragraph{Fokus auf die Gesamtbewertung}
Die erzielten Ergebnisse sind sich im Verhältnis zueinander sehr ähnlich, vor allem Google, Alibaba und AWS haben nahezu das gleiche Ergebnis erzielt.

\subsection{Bedeutung der Ergebnisse}
\paragraph{Niedrigste Anbieterbindung}
Es lässt sich den Ergebnissen entnehmen, dass, auf Basis der angewendeten Methodik, \glq Microsoft's Azure\grq\ die niedrigste Anbieterbindung erzeugt.
Das Ergbnis des Anbieters liegt jedoch trotzdem über 50\%.
Diese Zahl sagt aus, dass, aus technischer Sicht, etwa halb so viel Aufwand betrieben werden muss, 
um den Anbieter zu wechseln, als bei einer Migration, 
wo für jedes Produkt eine Alternative gefunden werden muss.

\subsection{Aussagekraft des Modells}
Die Ergebnisse fundieren auf der Annahme, dass eine Migration erleichtert wird, 
wenn das Migrationsziel und der -Ursprung die gleiche Technologie für die Anwendungen nutzen.
Des Weiteren wird angenommen, 
dass eine Dokumentation der Architektur und der Schnittstellen einer Anwendung nicht nur die Entwicklung und Arbeit während der Verwendung eines Cloud-Computing-Anbieters erleichtern,
sondern auch bei einer Migration.

\subsection{Diskussion der Methodik}
Bei dieser Herangehensweise werden Punkte wie die finanziellen Möglichkeiten des Unternehmens zum Zeitpunkt der Migration und die geschaffenene Rahmenbedingungen durch organisatorische Punkte wie SLAs ignoriert.
Dies wurde getan, weil das Thema Anbieterbindung vom Zeitpunkt der Anbieterwahl betrachtet werden sollte. 
Die anderen Komponenten der Anbieterbindung werden bei der Gründung einer Beziehung zwischen Cloud-Computing-Anbieter und Kunde relevant und, 
wenn der Wunsch zum Wechsel eines Anbieter feststeht.
