\chapter{Lösungsansatz}\label{ch:method}

\section{Bewertungsmodell}
Das Bewertungsmodell stellt im Hinblick auf die Zielsetzung dieser Arbeit die primär angesetzt Lösung dar, wobei zum Schluss zusätzliche Lösungen für Herausforderung der Anbieterbindung in Aussicht gestellt werden.

\subsection{Struktur des Kriterienkatalogs}
\paragraph{Überblick über die Struktur}
Die einzelnen Kriterien werden in Bewertungskategorien gegliedert.
Zudem hat der gesamte Katalog zwei Hauptbestandteile: Kategorien mit Kriterien (\gls{SaaS}-Produkte) mit indirektem und Kategorien mit direktem Einfluss auf die Anbieterbindung.
Aus diesem Ansatz ergibt sich die Hierarchie aus Abbildung \ref{bewertungs-kat-tree}

\begin{figure}[h]
\begin{forest}
[Bewertungsmodell, calign=center
[indirekter Einfluss
  [Business-SaaS
    [CRM]
    [ERP]
  ]
  [...]
  [Virtual-Desktop]
]
[direkter Einfluss
  [Interoperabilität
    [API]
  ]
  [...]
  [Containerisierung]
]
]
\end{forest}
\caption{Skizze der hierarchischen Struktur des Kriterienkatalogs}
\label{bewertungs-kat-tree}
\end{figure}

\paragraph{Motivation für diese Aufteilung}
Die Art der Aufteilung führt zu einem übersichtlichen Kriterienkatalog, da die einzelnen Bestandteile separat betrachtet und bei einer Untersuchung nacheinander ausgewertet werden können. 
Die aufgelisteten Kriterien lassen sich vom Anwender des Katalogs anhand der Informationen des Anbieter-Seite ausfüllen.

%Wie im Abschnitt \ref{produkt-übersicht-saas} auflistet, 
Es haben sich gängige Produkte etabliert und führende Anbieter bieten üblicherweise alle diese Produkte an.
Generell ist die Frage mit einem Blick auf die Produktkataloge der führenden und größten Anbieter eher wie genau die Kriterien umgesetzt wurden und nicht,
ob die Kriterien überfüllt wurden.
Es wird argumentiert, dass der definierte Lösungsansatz Anbieter, die Produkte mit besseren Migrationsvoraussetzungen anbieten, besser bewertet. Gestützt wird das, durch die Rechnung \ref{Minimal100}.

\subsection{Konkretisierung der Kriterien}
\label{kat-indirekt-lockin}
\paragraph{Entwicklungsprozess konkreter Kriterien}
Wie in Abschnitt \ref{entw-bewertungs-kat} beschrieben wurde folgende SaaS-Cloud-Anbieter untersucht:

\begin{itemize}
\item[-] Alibaba Cloud
\item[-] Amazon Azure
\item[-] Google Cloud Plattform
\end{itemize}

\paragraph{Handhabung der Bewertung}
In der Definition des Lösungsansatzes wurde formal beschrieben wie die Bewertung abläuft. Berücksichtigt wird die Existenz eines Produkt, die dazugehörige Dokumentation und die zugrundeliegende Technologie und, ob diese ein Standard beziehungsweise eine offene Technologie ist (vergleiche \ref{erw-kriterien-komponenten}).

\paragraph{Begründung dieser Vorgehensweise}
Produkte eines SaaS-Cloud-Anbieters hat einen indirekten Einfluss auf die Anbieterbindung, wenn ein konkretes Produkt Daten nutzt oder produziert, dessen Format umgewandelt werden muss oder die migriert werden müssen. 
Je mehr SaaS-Produkte ein direktes Pendant bei einem Ziel-Anbieter haben, desto schwächer ist die Anbieterbindung. 
Ein direktes Pedant ist ein Produkt, das bei dem Ziel- und Ursprungs-Anbieter auf der gleichen Technologie beziehungsweise dem gleichen offenen Standard funktioniert.
Nach dieser Logik ist die Bewertung sinnvoll, wenn Migration erleichtert werden soll. 
Migrationskomplexität ist ein erheblicher Faktor der Anbieterbindung.

\paragraph{Diskussion der Handhabung}
Der Entwicklungsprozess für den Kriterienkatalog verwendet eine Auswahl von Anbietern als Basis für den Katalog.
Der benötigte Umfang der Menge der Anbieter, die als Quelle für den Kriterienkataloges dienten, ist unbekannt.

Für diese Arbeit wurde mit vier Anbietern als Grundlage entwickelt, um die Zeit für die Recherche auf den Seiten der Anbieter im Vergleich zu allen anderen Recherchen und der tatsächlichen Ausarbeitung der Lösung zu reduzieren. Gleichermaßen wäre die Verwendung einer größeren Anbietermenge möglich gewesen.
 
\subparagraph{Gründe gegen eine größere Basismenge}
In Relation zu den im Kapitel \ref{lead-cc-A} aufgeführten Cloud-Anbietern, entspricht die Stichprobengröße von 4 rund 15\% der Gesamtzahl der größten Anbieter.

\begin{eqnarray}
a_{Seed} = \{Alibaba, Amazon, Google\}, |a| = 3
\end{eqnarray}
\begin{eqnarray}
A_{Gesamtanbietermenge} = \{...\}, |A| = 20
\end{eqnarray}
\begin{eqnarray}
a \subset A
\end{eqnarray}
\begin{eqnarray}
R_{Gesamtanbieterzahl} = \frac{|a|}{|A|}
\end{eqnarray}
\begin{eqnarray}
R_{Gesamtanbieterzahl} = \frac{3}{20} = 0,15 = 15%
\end{eqnarray}

Es werden zusätzliche Anbieter der gesamten Anbieter-Menge (vergleiche \ref{lead-cc-A}) für die Experimente mit der erarbeiteten Lösung benötigt,
die nicht für die Entwicklung verwendet wurden.
(vergleiche \ref{krit-kat-experimente})

Falls durch den verwendeten Entwicklungsprozess ein Bias zugunsten von Anbietern aus der Grundlage entstanden ist, lässt sich das gegebenenfalls mit den zusätzlichen Anbietern zeigen.

Eine ausführliche und systematische Untersuchung über die ausreichende Größe der Stichprobe ist nicht Gegenstand dieser Arbeit, dies würde unabhängige Entwicklung mehrerer Kataloge mit verschieden großen Basis-Anbietern benötigen.

\paragraph{Kriterien mit indirektem Einfluss}

\subparagraph{SaaS-Produkte für Tagesgeschäfte}
In dieser Bewertungskategorie wird der Anbieter auf seine Leistungen hinsichtlich Anwendungen für das Tagesgeschäft überprüft.
Darunter zählen hier Anwendungen zur Pflege der Kundenbeziehungen oder der Unternehmensressourcen (CRM und ERP)
Außerdem zählen Kommunikationsanwendungen in diese Kategorie.

\subparagraph{Saas-Produkte zur Datenspeicherung}
Es wird betrachtet welche Datenbank-Formate zum Einsatz kommen, um Daten des Unternehmens zu Speichern. Außerdem
werden andere Speicherlösungen wie Blob-Speicher betrachtet.

\subparagraph{Entwicklungs-Werzeuge}
Für Unternehmen, wie die kubus IT, in denen Anwendungen und Skripte für den eigenen Gebrauch innerhalb der Firma entwickelt werden,
sind Werkzeuge zur Zusammenarbeit an Entwicklungsprojekten, zum Teilen und Speichern von Programmcode im Sinne einer Versionsverwaltung und zum effizienten Schreiben von Code im Sinne einer Entwicklungsumgebung notwendig.

\subparagraph{Betriebswerkzeuge}
Produkte, die eine Bedienoberfläche zur Überwachung oder der Verwaltung von Ressourcenverbrauch. Konkret sind Ressourcen beispielsweise Budget für Berechnung durch KI, Netzwerkverkehr oder Rechenressourcen wie Speicher.
Darüber hinaus gibt es teilweise auch Produkte zum Auswerten von beispielsweise CO2-Emissionen.

\subparagraph{IT-Sicherheit}
Diese Kategorie beinhaltet SaaS-Produkte für Identitäten- und Zugriffsmanagement (IAM). Das Konzept für Rechte, Rollen und deren Inhaber wächst üblicherweise mit der Zeit und kann sehr komplex werden. Da es erstrebenswert ist, dass die Art und Weise wie Identitäten auf der Seite des Cloud-Anbieters funktionieren möglichst kompatibel mit anderen Lösungen ist, wird auch diese Kategorie im Katalog berücksichtigt.
Außerdem beinhaltet es SaaS-Produkte, die zur Stabilität im Betrieb beitragen, schützen Beispielsweise die Infrastruktur vor Angriffen. Auch hier kann eine Migration notwendig werden, wenn beispielsweise Firewall-Einstellungen bei dem neuen Anbieter übernommen werden sollen.

\subparagraph{Virtual-Desktop-Infrastruktur}
Betrifft Produkte eines Anbieters, die zum Betrieb und zur Steuerung von virtuellen Desktops und allgemeiner Endgeräten wie beispielsweise auch Smartphones notwendig sind. 

\paragraph{Kriterien mit direktem Einfluss}
\subparagraph{Interoperabilität}
In dieser Kategorie werden Kriterien gesammelt, die zur generellen Interoperabilität beitragen und sich keiner anderen Kategorie unterordnen lassen.

Das Konzept kann sowohl im Sinne der Zusammenarbeit zwischen mehreren aktiven Cloud-Anbietern innerhalb eines Unternehmen, der Zusammenarbeit von eigenen Rechenzentren und Cloud-Infrastruktur und auch im Sinne der Zusammenarbeit zwischen einem neuen und einem ehemaligen Anbieter verstanden werden.
Diese Bereiche werden alle durch Kriterien dieser Kategorie abgedeckt.

\subparagraph{Migrationswerkzeuge}
Werkzeuge zur Migration können als Werkzeuge, die beim Wechsel zu einem Anbieter hin oder von diesem weg, unterstützen verstanden werden. Darunter zählen zum Beispiel Anwendungen zum Export oder zum Versenden von gespeicherten Daten zu oder von einem anderen Anbieter.

\subparagraph{Container-Lösungen}
Containerisierung unterstützt beim Betreiben von verschiedenen Anwendungen, indem diese Anwendungen in definierte Umgebungen gepackt werden. 
Es wird untersucht welche Containerisierungstechnologien zum Einsatz kommen.

\subsection{Entwicklung des Bewertungsschemas}
\label{entw-bewertungs-kat}
\paragraph{Grundlagen}
Zur Ermittlung dieser Gesamtbewertung (G) des Risikos wird ein gewichtetes Scoring-Modell entwickelt. Es wird ein Katalog (K) mit Kriterien (k) aufgestellt:
\begin{eqnarray}
K = \{k_{1},k_{2},...,k_{n-1},k_{n}\},\ n \in \mathbb{N}
\end{eqnarray} 

Ein Kriterium kann entweder erfüllt oder nicht erfüllt sein. Zunächst ist die einzige Komponente für ein erfülltes Kriterium die Existenz eines Produktes von Seiten des Anbieters. Dadurch ist die Variable $x$ (kurz für: Existenz) mit zwei Zuständen versehen:
\begin{eqnarray}
x_{k} \in \{0,1\}
\end{eqnarray}

Die Gesamtbewertung (G) wird aus dem Verhältnis aller erfüllten Kriterien zu der maximal möglichen Punktzahl errechnet. Das Ergebnis wird in Prozent angegeben. Generell gilt: Je höher das Ergebnis, desto niedriger das Risko für Anbieterbindung.
\begin{eqnarray}
G_{ungewichtet} = \frac{\displaystyle\sum_{k \in K} ^{} x_{k}}{|K|}\cdot 100
\end{eqnarray}

\paragraph{Hintergrund der Gewichtungen}
Individuell zu jedem Kriterium wird ein Gewicht zugeordnet. Das Gewicht (w,englisch: weight) kann prinzipiell eine beliebige reelle positive Zahl oder Null sein. Es wird gefordert, dass die Summe aller Gewicht 1 ist (Normierung der Gewichte), was die spätere Errechnung Gesamtwertes vereinfacht:
\begin{eqnarray}
w_{k} \in \mathbb{R}^{+}_{0} mit \displaystyle\sum_{k \in K} ^{} w_{k} = 1
\end{eqnarray}
Für die Gewichtung wurde der englische Anfangsbuchstabe verwendet, um Verwechslung mit der Gesamtbewertung zu verhindern.

Standardmäßig sind alle Kriterien mit einfacher Gewichtung konfiguriert. Das Teilergebnis (g) für ein für ein gewichtetes Kriterium errechnet sich aus:
\begin{eqnarray}
g_{k} = w_{k}  \cdot x_{k} 
\end{eqnarray}

\paragraph{Feinjustierung durch Anwender}
Es ist nicht vorgesehen, dass die Inhalte vom Anwender des Scoring-Modells bearbeitet werden.
Falls einzelne Kriterien im Anwendungsszenario nicht relevant sein sollten, können diese Verworfen werden, indem die dazugehörige Gewichtung auf Null gesetzt wird.

Je nach Fokus und Bedürfnissen des Anwenders können die einzelnen Kriterien auch durch Gewichtung priorisiert werden.

Dadurch kann das breite Spektrum der Kategorien und Kriterien nach Bedarf reduziert werden.

Durch Anwenden der normierten Gewichte ergibt sich die Berechnung der Gesamtbewertung in Prozent als:
\begin{eqnarray}
G_{gewichtet} = \displaystyle\sum_{k \in K} ^{} w_{k}\cdot x_{k}\cdot 100
\end{eqnarray}

\paragraph{Erweiterung der Kriterienkomponenten}
\label{erw-kriterien-komponenten}
Über die reine Existenz eines Produktes hinaus, kann auf die Beschaffenheit des Produktes einberechnet werden-

Wenn das Modell nur zählt wie viele Kriterien erfüllt werden, ist die Bewertung mehr ein Indikator für die Vielfältigkeit und die Diversität des Produktkatalogs des Anbieters.
Es ist wichtig, dass ein Anbieter alle erforderlichen Produkte in der SaaS-Cloud anbietet, jedoch sind weitere Faktoren entscheidend:
\begin{itemize}
\item[-] Verwendung von Standardsoftware
\item[-] Verwendung von frei zugänglichen Software
\item[-] Dokumentation der Schnittstellen
\item[-] Dokumentation der Funktionsweise
\end{itemize}

Als frei zugänglich gilt eine Software, wenn diese unter einer Open-Source-Lizenz verteilt wird.
Diese Faktoren werden integriert. Die Variable $t$ (kurz für: Technologie) repräsentiert die Zustände interne, Standard- oder offene Technologie beziehungsweise Software.
\begin{eqnarray}
t_{k} \in \{0,1,2\}
\end{eqnarray}

Der Umfang der Dokumentation wird mit der Variable $d$ (kurz für: Dokumentation) und den zwei Zuständen vorhanden oder nicht vorhanden modelliert.
\begin{eqnarray}
d_{k} \in \{0,1\}
\end{eqnarray}

\paragraph{Integration der zusätzlichen Komponenten}
Die neuen Komponenten werden mit der bisherigen Existenz-Komponente kombiniert.
Eine Dokumentation bei einer Standard-Software nicht zwangsläufig notwendig ist,
da die Informationen bereits vom Hersteller geliefert werden.
Dieser Umstand wird bei der Verknüpfung berücksichtigt.
Punkte sollte als eine Situation geben,
wenn es sich um eine Standard-Software handelt oder eine Dokumentation existiert oder beides der Fall ist.
\begin{eqnarray}
g'_{k} = d_{k} + t_{k}
\label{s-strich-von-k}
\end{eqnarray}

Wenn kein Produkt zum Kriterium existiert, soll das Ergebnis immer null Punkte ergeben, daher gilt für ein Kriterium:
\begin{eqnarray}
g_{k} = w_{k}\cdot x_{k}\cdot(g'_{k})
\end{eqnarray}
Die gesamte Formel zur Errechnung der Bewertung lautet folglich:
\begin{eqnarray}
G_{gewichtet} = \displaystyle\sum_{k \in K} ^{} (w_{k}\cdot x_{k}\cdot(g'_{k}))\cdot 100
\label{Finale-Formel}
\end{eqnarray}

\paragraph{Effekt auf die minimalem Risiko}
Um die beste Bewertung beziehungsweise des niedrigste Risiko für Anbieterbindung zu berechnen,
muss für $g'_{k}$ (von \ref{s-strich-von-k}) das Maximum gewählt werden, was drei entspricht. 
Die beste Bewertung wird also der Produktkatalog eines Cloud-Computing-Anbieters bekommen,
der alle Kriterien mit ausschließlich frei zugänglicher Software ($t_{k} = 2$) und zusätzlicher Dokumentation ($d_{k} = 1$) anbietet.
Angenommen der vollständige Katalog umfasst beispielsweise 100 Kriterien,
so ist die maximal mögliche absolute Bewertung bei einheitlicher normierter Gewichtung 3:

\begin{eqnarray}
G_{Maximal100} = \displaystyle\sum_{k\in K} ^{} w_{k}\cdot x_{k}\cdot (d_{k} + t_{k}) = 
\displaystyle\sum_{k \in K} ^{} w_{k}\cdot 1\cdot (1+2) = 
1 \cdot  3 = 3
\end{eqnarray}

Dieses konkrete Beispiel lässt sich allgemein schreiben und es wird gezeigt sich, dass die maximale Bewertung unabhängig von der Kriterienzahl immer drei ist:

\begin{eqnarray}
G_{Maximal} = \displaystyle\sum_{k\in K} ^{} w_{k}\cdot x_{k}\cdot (d_{k} + t_{k}) = 
\displaystyle\sum_{k\in K} ^{} w_{k}\cdot 1\cdot (1 + 2) = 3 \cdot \displaystyle\sum_{k\in K} ^{} w_{k} = 3
\end{eqnarray}

\paragraph{Charakterisirung eines Anbieters mit maximalem Risko}
Ein Cloud-Computing-Anbieter,
der keine Open-Source- oder Standard-Software anbietet erhält bei einheitlich einfacher Gewichtung und 100 Kriterien keine Punkte:

\begin{eqnarray}
G_{Minimal100} = \displaystyle\sum_{k \in K} ^{} w_{k}*x_{k}*(d_{k} + t_{k}) = \displaystyle\sum_{k \in K} ^{} 1*1*(0) = 100 * 0 = 0
\label{Minimal100}
\end{eqnarray}

Ein Anbieter, der hier keine Punkte erhält, kann dennoch Produkte anbieten,
die mit anderen Eigenschaften beziehungsweise Qualitätsmerkmalen von Software überzeugen 
(beispielsweise durch ausgezeichnete Bedienbarkeit der Oberfläche, Sparsamkeit mit Rechenressourcen oder IT-Sicherheit). 
Diese anderen Qualitätsmerkmale sind jedoch nicht Gegenstand der Analyse.

\paragraph{Zusätzliche Eigenschaften des Modells}
Ein Anbieter, der keine Kriterien erfüllt erhält unabhängig von der Gewichtung keine Punkte.
Wenn ein Anwender des Modells alle Gewichte auf null setzt, dann ist die Gesamtbewertung ebenfalls null.
Theoretisch könnte ein Anwender eingeben,
dass der Anbieter das Kriterium nicht mit einem Produkt erfüllt,
aber das Produkt beispielsweise dennoch eine Dokumentation aufweist. Diese Eingabe ergibt keine Punkte.
Auch diese drei Möglichkeiten könnten als Beispiel für eine minimale Gesamtpunktzahl (vergleiche Formel \ref{Minimal100}) angeführt werden, sind jedoch in der Praxis uninteressant.

\paragraph{Umgang mit Gewichtung}
Das Bewertungsschema ist so konzipiert, dass durch Feinjustierung der Gewichte, die Bedürfnisse des Anwenders berücksichtigt werden können.
Initial sind jedoch folgende Gewichtungskonfigurationen vorgesehen.

\subparagraph{Einheitliche Gewichtung}
Jedes Kriterium hat das gleiche Gewicht. Das Endergebnis hängt alleine von der Zahl der erfüllten Kriterien ab.

\subparagraph{Schwerpunkt bei direken Kriterien}
Viele erfüllte direkte Kriterien, bedeuten dass Aufwand eines Anbieterwechsels auf technischer Ebene erleichtert, denn es werden viele Hilfestellungen bei der Migration geboten. 
Um einen Schwerpunkt an dieser Stelle zu setzen, 
werden die Gewichte der Kriterien um einen gewissen Faktor erhöht.
Alternativ kann das Bewertungsmodell auch so Konfiguriert werden, 
dass die übrigen Kriterien keine Gewichtung haben, 
wodurch die Gesamtbewertung ausschließlich direkte Kriterien beinhaltet.
 
\subparagraph{Schwerpunkt bei indirekten Kriterien}
Sind viele indirekte Kriterien erfüllt, so lassen sich viele alte Anwendungen zu einen neuen Anbieter umziehen, weil die verwendeten Anwendungen nicht exklusiv für einen Anbieter sind. 
Das erleichtert die Migration der Anwendungsdaten, 
da die Daten nicht in das Format neuer Anwendungen übersetzt werden müssen. 
Außerdem ist eine solche Migration für die Mitarbeitenden weniger spürbar und diese müssen nicht in neue Software eingearbeitet werden.
Daher haben auch die indirekten Kriterien eine große Relevanz. 
Analog zu dem vorherigen Paragraph, 
kann entweder ein alleinige Analyse oder schwerpunktmäßige Berücksichtigung 
durch eine entsprechende Konfiguration der Kriterien erfolgen.


\subsection{Implementierung in einer Tabellenkalkulationssoftware}
\paragraph{Auswahl der Software}
Die Implementierung des Lösungsansatzes kann prinzipiell in einer beliebigen Software für Tabellenkalkulation erfolgen.
Aufgrund der Tatsache, dass der IT-Dienstleister das Office-Software-Paket von \glq Microsoft\grq  verwendet, wurde das Programm \glq Microsoft Excel\grq\ ausgewählt.

\paragraph{Aufbau der Datei}
Es wurde eine leere Vorlage für anstehende Anbieter-Untersuchungen erstellt.
Die Implementierung des Bewertungsmodells besteht aus vier Bausteinen:
\begin{enumerate}
\item[-] Kopf mit Informationen über die aktuelle Analyse
\item[-] Kriterienkatalog mit Kontrollkästchen (englisch: Checkboxes) und Auswahllisten (englisch: Dropdown-Lists)
\item[-] Ergebnisbereich mit verschiedenen Metrik und der Gesamtbewertung in Prozent
\item[-] Hinweise zu Abkürzungen und Erklärungen
\end{enumerate}

Die leere Vorlage ist im Anhang dieser Arbeit.

\paragraph{Änderungen und Anpassungen}
Die Gesamtbewertung in Prozent wurde zur intuitiven Ablesung von 100\% abgezogen.
Ein Punkt des Zieles war eine Metrik zur Abbildung des Vendor-Lock-In-Risikos.
Jedoch fordern die Kriterien des Katalogs Merkmale, die sich, laut der Argumentation der Arbeit, negativ auf die Anbieterbindung auswirken. Eine höhere Abdeckung dieser Kriterien ist als die Bewertung für niedrigeres Risko zum Vendor-Lock-In.

Zusammenfassend wurde die Gesamtbewertung invertiert, um näher an die Formulierung des Ziels zu kommen. Dieses Umdrehen des Ergebnisses ist eine gestalterische Entscheidung.

Die praktischen Demonstrationen und Experimente mit dem Bewertungsmodell unterstreichen die theoretische Funktionsweise.

\paragraph{Erstellen von vergleichenden Analysen}
Aus dem Ergebnisbereich werden die Werte automatisch über die Abruf-Funktion von \glq Microsoft Excel\grq extrahiert und in einer anderen Tabellen-Datei eingefügt.
Hier werden Diagramme zur Gegenüberstellung der unterschiedlichen Cloud-Computing-Anbieter angefertigt.
 