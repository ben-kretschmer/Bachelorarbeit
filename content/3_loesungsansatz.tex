\chapter{Lösungsansatz}\label{ch:method}

\section{Scoringmodell}

\subsection{Definition des Lösungsansatzes}
Zur Ermittlung dieser Gesamtbewertung des Risikos wird ein gewichtetes Scoring-Modell entwickelt.
Die inhaltlichen Kriterien des Modells können unabhängig von den jeweils dazugehörigen Gewichtungen definiert werden. 
Abhängig von den Ergebnissen der Verifizierungsphase können die Gewichte im Nachgang angepasst werden.

Je nach Fokus und Bedürftnissen des Anwenders können die einzelnen Kategorien auch durch Gewichtung priorisiert werden.
Dadurch kann das breite Spektrum der Kategorien nach Bedarf reduziert werden.

\subsection{Entwicklung von Bewertungskategorien}
Die einzelnen Kriterien werden in Bewertungskategorien gegliedert.
Grundsätzlich wurden die Kategorien durch die Ermittlung der Produktschnittmenge der Anbieter festgelegt:

\begin{itemize}
\item[-] Amazon Azure
\end{itemize}

Darüberhinaus werden die Definitionen und Anforderungen folgender Stellen berücksichtigt:

\begin{itemize}
\item[-] Gematik GmbH
\item[-] Sozialgesetzbuch
\end{itemize}

\subsection{Generelle Gesichtpunkte}
Allgemein wird bei jedem Kriterium bewertet, ob der Cloud-Anbieter dieses Kriterium erfüllt.
Die ungewichtete Punktzahl wird verdoppelt,
wenn das Kriterium durch eine Standartlösung erfüllt wird, die nicht für den konkreten Anbieter einzigartig ist.

\paragraph{SaaS für Tagesgeschäfte}
In dieser Bewertungskategorie wird der Anbieter auf seine Leistungen hinsichtlich Anwendungen für das Tagesgeschäft überprüft.
Darunter zählen hier Anwendungen zur Pflege der Kundenbeziehungen oder der Unternehmensressourcen (CRM und ERP)
Außerdem zählen Kommunikationsanwendungen in diese Kategorie.

\paragraph{Datenbank-Lösungen}
Es wird betrachtet welche Datenbank-Formate zum Einsatz kommen, um Daten des Unternehmens zu Speichern.

\paragraph{Container-Lösungen}
Containerisierung unterstützt beim Betreiben von verschiedenen Anwendungen, indem diese Anwendungen in definierte Umgebungen gepackt werden. 
Es wird untersucht welche Containiersierungtechnologien zum Einsatz kommen.

\paragraph{Entwicklungs-Werzeuge}
Für Unternehmen, wie die kubus IT, in denen Anwendungen und Skripte für den eigenen Gebrauch innerhalb der Firma entwickelt werden,
sind Werkzeuge zur Zusammenarbeit an Entwicklungsprojekten, zum Teilen und Speichern von Programmcode im Sinne einer Versionsverwaltung und zum effizienten Schreiben von Code im Sinne einer Entwicklungsumgebung notwendig.

\paragraph{Interoperabilität}
In dieser Kategorie werden Kriterien gesammelt, die zur generellen Interoperabilität beitragen und sich keiner anderen Kategorie unterordnen lassen.
Unter Interoperabilität versteht man...

Das Konzept kann sowohl im Sinne der Zusammenarbeit zwischen mehreren aktiven Cloud-Anbietern innerhalb eines Unternehmen, der Zusammenarbeit von eigenen Rechenzentren und Cloud-Infrastruktur und auch im Sinne der Zusammenarbeit zwischen einem neuen und einem ehemaligen Anbieter verstanden werden.
Diese Bereiche werden alle durch Kriterien dieser Kategorie abgedeckt.

\paragraph{Indentitätenmanagment}
Das Konzept für Rechte, Rollen und deren Inhaber wächst üblich mit der Zeit und kann sehr komplex werden. Da es erstrebenswert ist, dass die Art und Weise wie Identitäten auf der Seite des Cloud-Anbieters funktionieren möglichst kompatibel mit anderen Lösungen ist, wird auch diese Kategorie im Katalog berücksichtigt.

\paragraph{Migrationswerkzeuge}
[…]

\paragraph{Operationsstabilität}
[…]

\paragraph{Überwachung und Management}
[…]

\paragraph{Speicher-Lösungen}     
[…]

\paragraph{Virtual-Desktop-Infrastruktur}
[…]

\subsection{Ausarbeitung der untergeordneten Kriterien}
[…]

\subsection{Umgang mit Gewichtung}
[…]

\section{Alternative Ansätze}

\subsection{tbd1}
[…]

\subsection{tbd2}
[…]
