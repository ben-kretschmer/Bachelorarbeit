\chapter{Lösungsansatz}\label{ch:method}

\section{Scoringmodell}

\subsection{Definition des Lösungsansatzes}
Zur Ermittlung dieser Gesamtbewertung des Risikos wird ein gewichtetes Scoring-Modell entwickelt.
Die inhaltlichen Kriterien des Modells können unabhängig von den jeweils dazugehörigen Gewichtungen definiert werden. 
Abhängig von den Ergebnissen der Verifizierungsphase können die Gewichte im Nachgang angepasst werden.

Je nach Fokus und Bedürftnissen des Anwenders können die einzelnen Kategorien auch durch Gewichtung priorisiert werden.
Dadurch kann das breite Spektrum der Kategorien nach Bedarf reduziert werden.

\subsection{Entwicklung von Bewertungskategorien}
Die einzelnen Kriterien werden in Bewertungskategorien gegliedert.
Grundsätzlich wurden die Kategorien durch die Ermittlung der Produktschnittmenge der Anbieter festgelegt:

\begin{itemize}
\item[-] Amazon Azure
\end{itemize}

Darüberhinaus werden die Definitionen und Anforderungen folgender Stellen berücksichtigt:

\begin{itemize}
\item[-] Gematik GmbH
\item[-] Sozialgesetzbuch
\end{itemize}

\paragraph{Angebotene SaaS-Applikationen}
[…]

\paragraph{Datenbank-Lösungen}
[…]

\paragraph{Container-Lösungen}
[…]

\paragraph{Entwicklungs-Werzeuge}
[…]

\paragraph{Interoperabilität}
[…]

\paragraph{Indentitätenmanagment}
[…]

\paragraph{Migrationswerkzeuge}
[…]

\paragraph{Operationsstabilität}
[…]

\paragraph{Überwachung und Management}
[…]

\paragraph{Speicher-Lösungen}     
[…]

\paragraph{Virtual-Desktop-Infrastruktur}
[…]

\subsection{Ausarbeitung der untergeordneten Kriterien}
[…]

\subsection{Umgang mit Gewichtung}
[…]

\section{Alternative Ansätze}

\subsection{tbd1}
[…]

\subsection{tbd2}
[…]
