\chapter{Lösungsansatz}\label{ch:method}

\section{Scoringmodell}

\subsection{Definition des Lösungsansatzes}
\paragraph{Grundlagen}
Zur Ermittlung dieser Gesamtbewertung (englisch: score) des Risikos wird ein gewichtetes Scoring-Modell entwickelt. Es wird ein Katalog (K) mit Kriterien (k) aufgestellt:
\begin{eqnarray}
K \in \{k_{1},k_{2},...,k_{n-1},k_{n}\},\ n \in \mathbb{N}
\end{eqnarray} 

Ein Kriterium kann entweder erfüllt oder nicht erfüllt sein:
\begin{eqnarray}
x_{k} \in \{0,1\}
\end{eqnarray}

Die Gesamtbewertung (S) wird aus dem Verhältnis aller erfüllten Kriterien zu der maximal möglichen Punktzahl errechnet.
\begin{eqnarray}
S_{ungewichtet} = \frac{\displaystyle\sum_{n=1} ^{|K|} x_{k}}{|K|}
\end{eqnarray}

\paragraph{Hintergrund der Gewichtungen}
Die Inhalte der Kriterien des Modells können unabhängig von den jeweils dazugehörigen Gewichtungen definiert werden. Das Gewicht (englisch: weight) kann prinzipiell eine beliebige rationale Zahl sein:
\begin{eqnarray}
w_{k} \in \mathbb{Q}
\end{eqnarray}

Standardmäßig sind alle Kriterien mit einfacher Gewichtung konfiguriert und dadurch einen Punkt wert. Die erreichte Punktzahl (s) für ein gewichtetes Kriterium errechnet sich aus:
\begin{eqnarray}
s_{k} = w_{k}  * x_{k} 
\end{eqnarray}

\paragraph{Feinjustierung durch Anwender}
Es ist nicht vorgesehen, dass die Inhalte vom Anwender des Scoring-Modells bearbeitet werden.
Falls einzelne Kriterien im Anwendungszenario nicht relevant sind, können diese Verworfen werden, indem die dazugehörige Gewichtung auf Null gesetzt wird.

Je nach Fokus und Bedürftnissen des Anwenders können die einzelnen Kriterien auch durch Gewichtung priorisiert werden.

Dadurch kann das breite Spektrum der Kategorien und Kriterien nach Bedarf reduziert werden.

Durch Anwenden der Gewichte ergibt sich die Berechnung der Gesamtbewertung:
\begin{eqnarray}
S_{gewichtet} = \frac{\displaystyle\sum_{n=1} ^{|K|} w_{k}*x_{k}}{\displaystyle\sum_{n=1} ^{|K|} w_{k}*1}
\end{eqnarray}

\subsection{Entwicklung von Bewertungskategorien}
Die einzelnen Kriterien werden in Bewertungskategorien gegliedert.
Grundsätzlich wurden die Kategorien durch die Ermittlung der Produktschnittmenge der Anbieter festgelegt:

\begin{itemize}
\item[-] Alibaba Cloud
\item[-] Amazon Azure
\item[-] Google Cloud Plattform
\item[-] IONOS Cloud
\end{itemize}

Darüberhinaus werden die Definitionen und Anforderungen folgender Stellen berücksichtigt:

\begin{itemize}
\item[-] Gematik GmbH
\item[-] Sozialgesetzbuch
\item[-] ISO-Norm 27001 und 50001
\item[-] C5-Standard
\end{itemize}

\subsection{Kategorien mit indirektem Einfluss}

\paragraph{Handhabung der Kategorien}
Allgemein wird bei jedem Kriterium bewertet, ob der Cloud-Anbieter überhaupt Produkte anbietet, die das jeweilige Kriterium erfüllen. Für jedes Kriterium ist dokumentiert, was ein adequates Produkt ausmacht.
Die ungewichtete Punktzahl wird verdoppelt,
wenn das Kriterium durch eine Standartlösung erfüllt wird, die demnach nicht für den konkreten Anbieter einzigartig ist.

\paragraph{Begründung dieser Vorgehensweise}
Die Kategorie hat einen indirekten Einfluss auf die Anbieterbindung, wenn ein konkretes Produkt Daten nutzt oder produziert, dessen Format umgewandelt werden muss oder die migriert werden müssen. 
Je mehr SaaS-Produkte ein direktes Pendant bei einem Ziel-Anbieter haben, desto schächer ist die Anbieterbindung. 
Ein direktes Pedant ist ein Produkt, das bei dem Ziel- und Ursprungs-Anbieter auf der gleichen Technologie beziehungsweise dem gleichen offenen Standard funktioniert.



\paragraph{Diskussion der Handhabung}


\paragraph{SaaS-Produkte für Tagesgeschäfte}
In dieser Bewertungskategorie wird der Anbieter auf seine Leistungen hinsichtlich Anwendungen für das Tagesgeschäft überprüft.
Darunter zählen hier Anwendungen zur Pflege der Kundenbeziehungen oder der Unternehmensressourcen (CRM und ERP)
Außerdem zählen Kommunikationsanwendungen in diese Kategorie.

\paragraph{Saas-Produkte zur Datenspeicherung}
Es wird betrachtet welche Datenbank-Formate zum Einsatz kommen, um Daten des Unternehmens zu Speichern. Außerdem
werden andere Speicherlösungen wie Blob-Speicher betrachtet.

\paragraph{Entwicklungs-Werzeuge}
Für Unternehmen, wie die kubus IT, in denen Anwendungen und Skripte für den eigenen Gebrauch innerhalb der Firma entwickelt werden,
sind Werkzeuge zur Zusammenarbeit an Entwicklungsprojekten, zum Teilen und Speichern von Programmcode im Sinne einer Versionsverwaltung und zum effizienten Schreiben von Code im Sinne einer Entwicklungsumgebung notwendig.

\paragraph{Indentitätenmanagment}
Das Konzept für Rechte, Rollen und deren Inhaber wächst üblich mit der Zeit und kann sehr komplex werden. Da es erstrebenswert ist, dass die Art und Weise wie Identitäten auf der Seite des Cloud-Anbieters funktionieren möglichst kompatibel mit anderen Lösungen ist, wird auch diese Kategorie im Katalog berücksichtigt.

\paragraph{Operationsstabilität}
SaaS-Produkte, die zur Stabilität im Betrieb beitragen, schützen Beispielsweise die Infrastruktur vor Angriffen. Auch hier kann eine Migration notwendig werden, wenn beispielsweise Firewall-Einstellungen bei dem neuen Anbieter übernommen werden sollen.

\paragraph{Überwachung und Management}
[…]

\paragraph{Virtual-Desktop-Infrastruktur}
Betrifft Produkte eines Anbieters, die 

\subsection{Kategorien mit direktem Einfluss}
\paragraph{Interoperabilität}
In dieser Kategorie werden Kriterien gesammelt, die zur generellen Interoperabilität beitragen und sich keiner anderen Kategorie unterordnen lassen.
Unter Interoperabilität versteht man...

Das Konzept kann sowohl im Sinne der Zusammenarbeit zwischen mehreren aktiven Cloud-Anbietern innerhalb eines Unternehmen, der Zusammenarbeit von eigenen Rechenzentren und Cloud-Infrastruktur und auch im Sinne der Zusammenarbeit zwischen einem neuen und einem ehemaligen Anbieter verstanden werden.
Diese Bereiche werden alle durch Kriterien dieser Kategorie abgedeckt.

\paragraph{Migrationswerkzeuge}
Werkzeuge zur Migration können als Werkzeuge, die beim Wechsel zu einem Anbieter hin oder von diesem weg, unterstützen verstanden werden. Darunter zählen zum Beispiel Anwendungen zum Export oder zum Versenden von gespeicherten Daten zu oder von einem anderen Anbieter.

\paragraph{Container-Lösungen}
Containerisierung unterstützt beim Betreiben von verschiedenen Anwendungen, indem diese Anwendungen in definierte Umgebungen gepackt werden. 
Es wird untersucht welche Containiersierungtechnologien zum Einsatz kommen.

\subsection{Umgang mit Gewichtung}
Die Lösung ist so konzipitiert, dass durch Feinjustierung der Gewichte, die Bedürftnisse des Anwenders berücksichtigt werden können.
Inital sind jedoch folgende Gewichtungskonfigurationen vorgesehen.
\paragraph{Einheitliche Gewichtung}
Jedes Kriterium hat das gleiche Gewicht. Das Endergebniss hängt alleine von der Zahl der erfüllten Kriterien ab.
\paragraph{Schwerpunkt bei direken Kriterien}
Viele erfüllte direkte Kriterien, bedeuten dass Aufwand eines Anbieterwechsels auf technischer Ebene erleichtert, denn es werden viele Hilfestellungen bei der Migration geboten. Um einen Schwerpuntk an dieser Stelle zu setzen, werden die Gewichte der Kriterien um einen gewissen Faktor erhöht.
Alternativ kann das Bewertungsmodell auch so Konfiguriert werden, dass die übrigen Kritieren keine Gewichtung haben, wodurch die Gesamtbewertung ausschließlich direkte Kriterien beinhaltet.
 
\paragraph{Schwerpunkt bei indirekten Kriterien}
Sind viele indirekte Kriterien erfüllt, so lassen sich viele alte Anwendungen zu einen neuen Anbieter umziehen, weil die verwendeten Anwendungen nicht exklusiv für einen Anbieter sind. Das erleichtert die Migration der Anwendungsdaten, da die Daten nicht in das Format neuer Anwendungen übersetzt werden müssen. Außerdem ist eine solche Migration für die Mitarbeitenden weniger spürbar und diese müssen nicht in neue Software eingearbeitet werden.
Daher haben auch die indirekten Kritieren eine große Relavanz. Analog zu dem vorherigen Paragraph, kann entweder ein alleinige Analyse oder schwerpunktmäßige Berücksichtigung durch eine entsprechende Konfiguration der Kriterien erfolgen.
