\chapter{Problemstellung}\label{ch:problemstellung}

\section{Ausgangssituation}

\subsection{Besonderheiten beim IT-Dienstleister kubus IT}
Die kubus IT ist durch ihre Funktion als Dienstleister für die gesetzlichen Krankenkassen AOK Bayern und AOK PLUS teil der öffentlichen Verwaltung.

Durch diese Position ergeben sich rechtliche Besonderheiten im Vergleich zu einem vergleichbaren Dienstleister der deutschen Wirtschaft.

[...]

\subsection{Vergabeverfahren im Vendor-Management}
Der Abteilung Einkauf ist die Abteilung Vendor-Management zur Orchestrierung der Beziehungen zwischen der kubus IT und Anbietern.

Die Anbieter werden in drei Kategorien segmentiert.

\begin{itemize}
\item[-] A-Vendoren: Große Abhängigkeit, hohe Kritkalität und fehlende kurzfristige Austauschbarkeit
\item[-] B-Vendoren: Mittlere Vorlaufeiten und Kosten für den Austausch
\item[-] C-Vendoren: Anbieter für Standardleistungen mit leichter Austauschbarkeit
\end{itemize}

Diese Anbieterkategorien sind nicht ausschließlich für Anbieter von Cloud-Computing konzepiert. Stattdessen wird bei jeder Geschäftsbeziehung im Vendoren-Management mit diesem Schema gearbeitet.

Demnach zählen folgende Anbieter aktuell in das A-Segment:

\begin{itemize}
\item[-] Arvato: Cloud-Computing
\item[-] DATAGROUP: 'IMAC'
\item[-] Avaya: Cloud-Kommunikation
\end{itemize}

\subsection{Bisherige Wechsel von Anbietern}

[…]

\section{Anforderungen}

\subsection{Zielsetzung der Analyse}
Das Analyseverfahren soll einen Ausgabewert liefern, der beschreibt wie hoch das Risiko eines Vendor-Lock-Ins ist.
Da diese Gesamtbewertung aus allen einzelnen Bewertungen aufsummiert wird, liegt es nahe das Risiko mit einer Prozentzahl zu bewerten. Die Schranken zur Ablehnung oder Akzeptierung eines Anbieter lassen sich dann wiederrum variabel wählen.

Hierbei ist eine strengere oder lockerere Hürde denkbar. 

Unabhängig von der Bewertung des Vendor-Lock-In-Risikos, ist dieser Punkt auch nur ein kleiner Bestandteil im gesamten Vergabeverfahren und andere wirtschaftlichen, strategischen und inhaltlichen Faktoren sind ebenfalls zu betrachten.

Daher kann die Gewichtung des Faktors Vendor-Lock-In-Risko im Gesamtbild dem jeweiligen Entscheidungsträger des Verfahrens überlassen werden.

Unabhängig von diesen Punkten ist die Gesamtbewertung des Risikos ein Werkzeug zum fairen Gegenüberstellen verschiedener Anbieter.

\subsection{Verifizierbarkeit der Ergebnisse}
Die Verifizierbarkeit lässt sich durch drei Methoden abbilden.
Es ist einerseits möglich die Bewertung des Modells mit der Einschätzung eins Experten im Cloud-Bereich zu vergleichen und andererseits bestehende Analysen als Messlatte zu wählen.

Darüberhinaus lassen sich die Ergebnisse auch durch das Messen des Aufwands der Migration testen.
Eine Metrik für den Aufwand wäre die benötigte Arbeitzeit, die Kosten der Migration oder der Anteil der Anwendungen, die umgestellt werden müssen.

Um die durch das Bewertungsmodell vorgeschlagene Bewertung per Experiment zu überprüfen, muss der Aufwand Emigration vom Anbieter betrachtet werden.
 Doch auch das Ziel des gesamten Migrationsprozesses ist hierbei relevant und hat einen Einfluss auf den gesamten Aufwand.
Generell werben Anbieter mit Angeboten und Werkzeugen, die bei der Migration unterstützen.

Damit die Experimente verschiedener Anbieter vergleichbar sind, sollte jedoch der Anteil des Aufwands, der auf die Immigration zurückzuführen ist, minimiert werden.
Dazu zählt auch die Erleichterung der Migration durch eventuelle Dienstleistungen des Ziel-Anbieters.

Zur Bewerk\-stelligung hiervon kann die Im\-migration in ein niederes Liefer\-modell gewählt werden.
So kann beispielsweise der Wechsel verschiedener Anbieter im Liefermodell Software-as-a-Service zu einem Anbieter mit dem Liefermodell Platform-as-a-Service betrachtet werden.

Obwohl der Trend zu umfangreicheren Liefermodellen geht und solche Migrationen generell eher aufwendig sind, hat diese Methode Relevanz, um die Bewertung der Anbieter generell zu überpüfen. Da solche Migrationen aufwendig sind, ist anzunehmen, dass die Aufwandsunterschiede der einzelen Anbieter in Relation zum Gesamtaufwand niedrig sind.

Auch mit der Migration innerhalb eines Liefermodells kann experimentiert werden.
Im Gegensatz zur vorherigen Methode ist anzunehmen, dass die Aufwandsunterschiede zwischen Anbietern innerhalb eines Liefermodells in Relation zum Gesamtaufwand drastischer sind. 
Wichtig ist jedoch für einen fairen Vergleich, dass die Vendor-Lock-In-Risikobewertung vom Wechselziel besser ist als die beste Bewertung der zu untersuchenden Menge an Anbietern.

Der Grund für diese Einschränkung ist einerseits die Notwendigkeit einer Verbesserung, denn obwohl es beispielsweise wirtschaftliche Argumente für einen Anbieter mit einer höheren Anbieterbindung geben kann, ist im Kontext dieser Arbeit ein Wechsel in eine stärkere Anbieterbindung generell nicht sinnvoll. 
Andererseits sorgt eine bessere Bewertung implizit für eine höhere Kompatibilität zwischen zwei Anbietern.


 



