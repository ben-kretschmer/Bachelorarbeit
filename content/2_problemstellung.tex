\chapter{Problemstellung}\label{ch:problemstellung}

\section{Ausgangssituation}
Der Abschnitt schlüsselt für den weiteren Verlauf der Arbeit den berücksichtigten Kontext im Sinne des Praxispartners kubus IT eGbR auf.

\subsection{Einordnung der Firma kubus IT}
\label{vorstellung-aok}
Diese Arbeit ist in Zusammenarbeit mit dem IT-Dienstleister kubus IT entstanden. 
Die Firma steht repräsentativ für ein Unternehmen Markt der gesetzlichen Krankenkassen.

\paragraph{Gesetzliche Krankenversicherungen} 
Das fünfte Sozialgesetzbuch (SGB) definiert, welche Institutionen gesetzliche Krankenversicherungen anbieten, wie diese Versicherungen ausgelegt sein müssen und welche Regeln eine gesetzliche Krankenkasse befolgen muss.
\\
Zu den beschrieben Krankenkassen zählen unter anderem die allgemeinen Ortskrankenkassen (\gls{AOK}), welche typischerweise in Regionen agieren, die nicht zwangsläufig den 16 Bundesländern entsprechen. \parencite[§143]{SGBV1988}
\\
Exemplarisch dafür steht die \gls{AOK} PLUS, welche in den Bundesländern Sachsen und Thüringen aktiv ist. Im Gegensatz dazu steht zum Beispiel die AOK Bayern, die lediglich in Bayern Krankenversicherungen anbietet.
\\
Am relevantesten für diese Arbeit ist jedoch, wie im Gesetzestext geregelt wird, welche konkreten Eigenschaften die IT-Systeme einer gesetzlichen Krankenkasse haben müssen. 

\paragraph{Verbindungen zwischen kubus IT und AOK}
Die Versicherungsunternehmen haben unterschiedliche Herangehensweisen für die Verwaltung der IT.
Dabei ist die kubus IT eine Arbeitsgemeinschaft der \gls{AOK} Bayern und \gls{AOK} PLUS.
In der Abbildung \ref{relation-it-service-tree} ist diese Beziehung in einer Skizze visuell dargestellt. Der \gls{IT}-Dienstleister übernimmt Beratung, Support, Entwicklung und Betrieb im Kontext IT für die beiden AOKen. Dabei wird des Weiteren dargestellt, dass die kubus IT von zusätzlichen Dienstleistern unterstützt wird. 
Die rechtlichen Vorgaben die für die Krankenkassen gelten müssen auch von der kubus IT umgesetzt werden.

\begin{figure}[h]
\begin{center}
\begin{forest}
[kubus IT,grow=north
[Data Group]
[Arvato]
[Microsoft]]
\end{forest}
\begin{forest}
[kubus IT
[AOK Bayern]
[AOK PLUS]]
\end{forest}
\end{center}
\caption{Skizze der Beziehung zwischen AOK Bayern, AOK PLUS und den IT-Dienstleistern}
\label{relation-it-service-tree}
\end{figure}

\subsection{Rechtliche Vorschriften für die IT}
\label{legal}
Relevant sind Kapitel zehn bis zwölf des fünften Sozialgesetzbuches, wobei zum Teil auch Paragraphen vorheriger Kapitel referenziert werden.
Die Vorschriften bieten den Raum für die Ausarbeitung von Lösungen hinsichtlich der Architektur der IT-Systeme.

\paragraph{Kapitel 10 - Buch V - Sozialgesetzbuch}
Das zehnte Kapitel des fünften Buches definiert Versicherungs- und Leistungsdaten, den Datenschutz und die Datentransparenz. 
Dieses Kapitel befasst sich mit Informations\-management. Ob die Informationen in digitaler Form oder in Form analoger Akten gespeichert sind ist hier nicht relevant. 
Stattdessen wird geregelt an wen die Daten weitergegeben oder durch wen die Daten verarbeitet werden dürfen. Auch Fristen und Regeln zur Vernichtung der Informationen wird aufgeführt. 
Diese Regelungen bilden indirekt Anforderungen an das Informations\-sicherheits\-management und die Zugriffs\-möglichkeiten auf Daten bei dem Cloud-Anbieter. \parencite[§284-§305b, Kap.~10]{SGBV1988}

\paragraph{Kapitel 11 - Buch V - Sozialgesetzbuch}
Das elfte Kapitel des fünften Buches regelt alle Belange der sogenannten Telematikinfrastruktur.
Laut dem §306 des selbigen Abschnitts handelt es sich bei der Telematikinfrastruktur um "die interoperable und kompatible Information-, Kommunikations- und Sicherheitsinfrastruktur, die der Vernetzung von Leistungserbringern, Kostenträgern, Versicherten und weiteren Akteuren des Gesundheitswesens sowie der der Rehabilitation und der Pflege dient und […]". \parencite[§306-§383, Kap.~11]{SGBV1988}

\subparagraph{Folgen des Paragraph 307}
In diesem Paragraphen werden datenschutzrechtliche Verantwortlichkeiten festgelegt (§307).
Dieser Paragraph legt zusammengefasst fest, dass die Nutzer einer verteilten Infrastruktur, also beispielsweise die Nutzer eines Cloud-Anbieters, für eine rechtskonforme sichere Verarbeitung der schützenswerten Daten innerhalb des Netzes der verteilten Infrastruktur verantwortlich sind.
Des Weiteren erstreckt sich die Verantwortlichkeit auch auf die ordnungsgemäße Inbetriebnahme, Wartung und Verwendung der Komponenten. Der Nutzer ist dann nicht verantwortlich,
wenn dieser nicht über die \glqq Mittel der Datenverarbeitung mitentscheiden\grqq\ 
\parencite[§307]{SGBV1988} kann,
wie es in §307 lautet. 
Diese Regelung bedeutet im Kontext der Cloud,
dass bereits beim Liefermodell Infrastructure-as-a-Service die Verantwortung zur rechtskonformen sicheren Verarbeitung beim Anbieter liegt,
denn schon ab diesem Niveau wird die Mitsprache durch den Nutzer der Cloud über die Mittel der Datenverarbeitung eingeschränkt.
Spätestens bei dem Liefermodell Software-as-a-Service, wo der gesamte Technologie-Stapel (einschließlich der Software) durch den Anbieter betrieben wird, liegt auch die Verantwortung vollständig bei diesem.

\subparagraph{Folgen des Paragraph 308}
In diesem Paragraphen wird der Vorrang von Schutzmaßnahmen vor der europäischen Datenschutz-Grundverordnung geregelt (§308).
Zusammenfassend werden die \glqq Rechte der betroffenen Person nach den Artikeln 12 bis 22 der Verordnung (EU) 2016/679 […] ausgeschlossen, soweit diese Rechte […] nicht oder nur unter Umgehung von Schutzmechanismen […] gewährleistet werden können.\grqq\ \parencite[§308]{SGBV1988}
Folglich muss ein Verantwortlicher einem Betroffenen Daten nicht aushändigen, wenn für die Aushändigung Sicherheitsmaßnahmen umgangen oder Daten ungesichert separat gespeichert werden müssten.
Dieses Gesetz widerspricht also zunächst dem allgemeinen europäischen Recht, dass jederzeit die Daten eines Betroffenen beispielsweise für diesen offengelegt oder gelöscht werden müssen, wenn dieser das verlangt.
Jene Reglung des §308 gilt allerdings auch nur dann, wenn die Datenverarbeitung rechtmäßig ist
und beispielsweise die Einsichtnahme zweifelsohne nicht ohne Umgehung von Sicherheitsvorkehrungen möglich ist. \parencite[§308]{SGBV1988}
Für den Cloud-Anbieter bedeutet das, dass eine Schnittstelle zur Verfügung gestellt werden sollte, die eine geregelte Einsichtnahme in die Daten eines Betroffenen ermöglicht. 


\paragraph{Kapitel 12 - Buch V - Sozialgesetzbuch}
Das zwölfte Kapitel beinhaltet unter anderem explizit den Einsatz von Cloud-Technologie im Gesundheitswesen in §393.
Dort wird festgelegt, dass die Verarbeitung von Sozial- und Gesundheitsdaten im Weges des Cloud-Computing-Dienstes nur im Inland, einem Mitgliedsstaat der Europäischen Union oder einem der Mitgliedstaat der europäischen Union gleichgestellten Staat wie der Schweiz und Mitgliedstaaten des europäischen Wirtschaftsraumes erfolgen darf. Alle anderen Staaten gelten als Drittstaaten müssen in einem sogenannten Angemessenheitsbeschluss für die Übermittlung personenbezogener Daten erst genehmigt werden. \parencite[§384-§395, Kap.~12]{SGBV1988}

Derzeit sind folgende Staaten als Drittstaat durch einen Angemessenheitsbeschluss zugelassen:
\begin{itemize}
\item[-] Andorra
\item[-] Argentinien
\item[-] Kanada
\item[-] Färöer-Inseln (Bestandteil des Königreichs Dänemark) \parencite{Faeroeer2026}
\item[-] Guernsey (Im britischen Kronbesitz) \parencite{Guernsey2026}
\item[-] Israel
\item[-] Isle of Man (Im autonomen britischen Kronbesitz) \parencite{IsleOfMan2026}
\item[-] Japan
\item[-] Jersey
\item[-] Neuseeland
\item[-] Republik Korea (Südkorea)
\item[-] Schweiz
\item[-] Uruguay
\item[-] Vereinigtes Königreich
\item[-] Vereinigte Staaten von Amerika
\end{itemize}
\label{geduldete-Länder}
\parencite{AngembeschlEU}

Darüber hinaus wird im Absatz 3 des §393 festgelegt, dass eine Verarbeitung nur zulässig ist, wenn technische und organisatorische Maßnahmen, die dem Stand der Technik entsprechen, zur Gewährleistung der Informationssicherheit ergriffen worden sind. Zudem muss ein Sicherheitszertifikat der datenverarbeitenden Stelle, also des Cloud-Anbieters, vorliegen und zusätzlich die Kriterien für Kunden, die im Prüfungsbericht des Testats enthalten sind, umgesetzt wurden. \parencite[Kap.~12]{SGBV1988}

Das geforderte Testat muss den Anbieter insofern zertifizieren, dass Sicherheitsvorkehrungen mit dem Niveau des C5-Typ2-Testats oder höher getroffen wurden.

Auf Basis dieser Gesetzte wird eine übergreifend gültige Strategie vom \gls{AOK} Bundesverband veröffentlicht.

\paragraph{Schlussfolgerungen für die Anbieterwahl}
Auf Grundlage dieser Gesetze werden Anbieter nach der Analyse aussortiert.
Die Analyse der ungeeigneten Anbieter wird zur Vollständigkeit trotzdem durchgeführt.

\subsection{Cloud-Architektur in der kubus IT}
Die Cloud-Architektur der kubus IT basiert auf den Vorgaben des AOK Bundesverbandes.
Die kubus IT verwendet das Modell der hybriden und teilt den Bedarf für Cloud-Computing in zwei Bereiche.
\paragraph{Personenbezogene Daten in der Private Cloud}
Die Private Cloud ist \glqq eine Cloud-Umgebung, die nur einen bestimmten geschlossenen Nutzerkreis (z.B. Mitarbeitende eines Unternehmens) zur Verfügung steht.\grqq \parencite[S.~42]{Hennrich2023}
Eine private Cloud-Umgebung nutzt dedizierte Infrastruktur was die Überwachung der Einhaltung der Gesetze zum Datenschutz erleichtert. 
Die Firma Arvato Systems ist der Cloud-Computing-Anbieter, der die private Cloud der AOK Bayern und AOK PLUS betreibt. Die Auslagerung des Betriebs der Infrastruktur ermöglicht hier einen Fokus und eine Spezialisierung auf beiden Seiten. So kann sich die kubus IT als IT-Dienstleister nach der Auslagerung auf den Betrieb der Anwendungen fokussieren.

\paragraph{Technische Daten in der Public Cloud}
Die Public Cloud ist \glqq eine Cloud-Umgebung, die von der Allgemeinheit \- also von \'jedermann\' \- genutzt werden kann.\grqq  \parencite[S.~39]{Hennrich2023}
Dennoch lassen sich die Daten und Dienste der Cloud durch Verschlüsselung und Authentifizierung sichern.
Der Nachweis der Einhaltung der gesetzlichen Vorgaben zum Schutz personenbezogener Daten ist jedoch schwieriger.
Die öffentliche Cloud beinhaltet stärker als die private Cloud die Skalierbarkeit und Agilität,
die generell Cloud-Computing zugeschrieben wird. 
An dieser Stelle wird aus Leistungs-Perspektive von den großen Spielern des Cloud-Computing profitiert werden.
Beispielsweise besonders deutlich wird das im Bereich der Entwicklung neuer Anwendungen durch den Dienstleister kubus IT.
Hier wird die Applikationen für Entwicklung und Betrieb von Microsoft verwendet (Microsoft Azure DevOps).
Dieses Produkt ist unter Anderem so attraktiv, weil die Integration mit der Microsoft-eigenen Entwicklungsumgebung (IDE) ausgezeichnet ist.
Außerdem bietet Azure DevOps viele Werkzeuge zur Umsetzung sauberer Entwicklungs- und Betriebsprozesse.

\subsection{Vendor-Management in der kubus IT}
\paragraph{Vorstellung der Abteilung Einkauf}
Der Abteilung Einkauf ist die Abteilung Vendor-Management zur Orchestrierung der Beziehungen zwischen der kubus IT und Anbietern untergeordnet.

Die Anbieter werden in drei Kategorien segmentiert.

\begin{itemize}
\item[-] A-Vendoren: Große Abhängigkeit, hohe Kritikalität und fehlende kurzfristige Austauschbarkeit
\item[-] B-Vendoren: Mittlere Vorlaufzeiten und Kosten für den Austausch
\item[-] C-Vendoren: Anbieter für Standardleistungen mit leichter Austauschbarkeit
\end{itemize}

Diese Anbieter-Kategorien sind nicht ausschließlich für Anbieter von Cloud-Computing konzipiert. Stattdessen wird bei jeder Geschäftsbeziehung im Vendoren-Management mit diesem Schema gearbeitet.

Demnach zählen folgende Anbieter aktuell in das A-Segment:

\begin{itemize}
\item[-] Arvato: Cloud-Computing
\item[-] DATAGROUP: 'Install-Manage-Add-Change'-Dienstleister
\item[-] Avaya: Cloud-Kommunikation
\end{itemize}

\parencite{vergabVendorManage2022}

\paragraph{Relevanz für die Arbeit}
Das Vendoren-Management ist verantwortlich für die Vergabe von Aufträgen an externe Dienstleister und folglich auch für die Wahl der Cloud-Computing-Anbieter. Daher werden diese Verantwortlichen im Kontext der Arbeit hier kurz vorgestellt.

\section{Zielsetzung}
In der Zielsetzung werden im Rahmen der Arbeit die Anforderungen an den Lösungsansatz, der auf Basis der etablierten Grundlage aus der Einleitung und der Ausgangssituation entwickelt wird, definiert.

\subsection{Analyse des Vendor-Lock-In-Riskos}
\paragraph{Hauptziel der Arbeit}
Das Ziel, auf das in dieser Arbeit hingearbeitet wird, ist die Analyse von führenden Cloud-Computing-Anbietern 
hinsichtlich ihrer Anbieterbindung oder dem Risiko für \glq Vendor-Lock-In \grq.
Die einzelnen Untersuchungen der Anbieter sollen dabei innerhalb von kurzer Zeit (wenige Stunden) durchgeführt werden.
Es werden keine konkretere Ziele bezüglich der Dauer oder des Umfanges vorgegeben.

\subparagraph{Ablauf der Analyse}
Für die Analyse ist zunächst die Entwicklung einer Metrik zur Gegenüberstellung von Anbietern hinsichtlich der Anbieterbindung notwendig.
Daraufhin wird die Bewertung der als führend ermittelten Anbieter stattfinden.
Nach der Bewertung werden die Ergebnisse dargestellt und mit verglichen.
Aus den Ergebnissen und der Gegenüberstellung wird eine Interpretation der aktuellen Situation gefolgert.

\subparagraph{Thematik Anbieterbindung}
Gegenstand der Analyse, also des Vergleichs ist die das Thema der Anbieterbindung. Die Herausforderung liegt darin, das Thema messbar zu machen.
Dabei wird es versucht eine Verbindung zwischen technischen Kriterien und dem Risiko für Anbieterbindung zu ermitteln.
Dies geschieht entgegen der Annahme, das Anbieterbindung ein rein organisatorisches Problem ist. (vergleiche Abschnitt \ref{anbieterbindung})

\subparagraph{Herstellung der Vergleichbarkeit}
Die führenden Cloud-Computing-Anbieter weisen umfangreiche Produktkataloge auf. Einzelne Produkte existieren bei nur einem Anbieter als Alleinstellungsmerkmal.
Das entwickelte Bewertungsschema soll die Vergleichbarkeit mit möglichst universell präsenten SaaS-Produkten herstellen.

\subparagraph{Gewünschter Effekt}
Im Abschnitt \ref{komponenten-anbieterbindung} werden die Komponenten der Anbieterbindung nach der Argumentation der Arbeit aufgezeigt.
Der letztliche Effekt, der durch die Herstellung der Vergleichbarkeit erzielt werden soll, ist die Wahl eines Anbieters mit Berücksichtigung der Interoperabilität zur Vorbeugung von Anbieterbindung.
Dieser Umstand wird durch eine eigene Darstellung unterstrichen (vergleiche Abbildung \ref{fig:drei-komponenten}).
Die Grafik hat zwei Elemente. 
Auf der linken Seite befindet sich eine Skala der Farbgebung für Einordnung des Risikos mit einem Verlauf von geringer bis zu hoher Anbieterbindung.
Diese Farbtöne werden in der rechten Seite aufgegriffen.
Das rechte Element der Grafik zeigt ein Dreieck, wobei die Ecken mit \glq technische Interoperabilität(1)\grq , \glq organisatorischer Rahmen (2)\grq\ und \glq finanziellem Spielraum (3)\grq\ beschriftet sind. 
Gemeint sind die Komponenten der Anbieterbindung. Im Zentrum befindet sich ein tief-rot gefärbtes Kreuz, 
dass eine Situation kennzeichnet, 
die weder durch vorteilhafte Interoperabilität oder einem organisatorischen Rahmen, 
noch durch einen ausreichenden finanziellen Spielraum geprägt ist. 
Der Farbton bedeutet entsprechend und der Argumentation der Arbeit eine hohe Anbieterbindung.
Im Kontrast hierzu steht ein hell-rotes Kreuz, analoger Begründung nach symbolisch für eine niedrigere Anbieterbindung. 
Das hell-rote Kreuz ist im Eck mit der Nummer eins.
Ein Pfeil zeigt vom tief-roten Kreuz zum hell-roten Kreuz. 
Im Kontext der Argumentführung der Arbeit soll diese Grafik unterstreichen, 
dass ein Fokus auf technische Interoperabilität einen signifikanten Einfluss hin zu einer niedrigeren Anbieterbindung haben kann.
Die Grafik enthält keine Skala und dient nur zur visuellen Untermauerung der Argumentation.
\begin{figure}
\begin{center}
\includegraphics[width=0.75\linewidth]{figures/Anbieterbindung-drei-Einflussfaktoren.drawio}
\end{center}
\caption{Eigene Darstellung über den vorgeschlagenen Effekt vom Fokus der technischen Interoperabilität.}
\label{fig:drei-komponenten}
\end{figure}
\subsection{Verifizierbarkeit der Ergebnisse}
Nachdem ein Prozess entwickelt wurde, der eine Bewertung des Vendor-Lock-In-Riskos liefert, folgt die Prüfung des Lösungsansatzes.

\paragraph{Vorstellung von Verifizerungsoptionen}
Vor der Entwicklung des Prozesses werden daher direkt Methoden der Verifikation vorgestellt.
Die Argumentation dieser Arbeit läuft darauf hinaus, dass die Verifizierbarkeit sich durch drei Methoden abbilden lässt.
Es ist einerseits möglich die Bewertung des Modells mit der Einschätzung eins Experten im Cloud-Bereich zu vergleichen und andererseits bestehende Analysen als Messlatte zu wählen.

Darüber hinaus lassen sich die Ergebnisse auch durch das praktische Messen des Aufwands der Migration testen.
Eine Metrik für den Aufwand wäre die benötigte Arbeitszeit, die Kosten der Migration oder der Anteil der Anwendungen, die umgestellt werden müssen.

Es lassen sich für alle drei Ansätze Probleme identifizieren.

\paragraph{Herausforderungen von Experimenten}
Um die durch das Bewertungsmodell vorgeschlagene Bewertung per Experiment zu überprüfen, muss der Aufwand Emigration vom Anbieter betrachtet werden.
Doch auch das Ziel des gesamten Migrationsprozesses ist hierbei relevant und hat einen Einfluss auf den gesamten Aufwand.
Generell werben Anbieter mit Geldprämien, Expertise und Werkzeugen, die bei der Immigration unterstützen. Beispielhaft dafür steht der Cloud-Computing-Anbieter Amazon mit dem \glq AWS Migration Acceleration Program\grq ,
welches unter Anderem aus einer \glqq flexiblen Migrationsmethode, 
die bewährte Frameworks nutzt\grqq , \glqq Tools, die helfen, 
Kosten zu reduzieren\grqq\ und \glqq AWS-Investitionen in Form von AWS-Service-Guthaben oder Partnerinvestitionen\grqq\ besteht. \parencite[AWS Migration Acceleration Program]{AWSMAP2026}

Migrationswerkzeuge als angebotene Plattform innerhalb des Produkt-Katalogs der Cloud-Computing-Dienstleister ist ein Kriterium des Bewertungsschemas.
Jedoch sind diese Unterstützungsangebote eine Einbahnstraße und die Anbieter werben meist ausdrücklich nur für die Hilfe beim Umzug der Daten auf die eigenen Systeme. Aus diesem Grund eignet sich der Vergleich der Unterstützungsangebote nicht, um eine zweite vergleichende Rangordnung aufzustellen und das entwickelte Modell zu prüfen.

Des Weiteren ist für diese Arbeit die Durchführung von realistischen experimentelle Migrationen nicht umsetzbar, 
da hierfür tatsächliche Geschäftsbeziehung zu den Anbietern zustande kommen müssten und sich eine tiefschichtige Anbieterbindung erst über eine längere Dauer der Beziehung entwickelt. (vergleiche Abbildung \ref{fig:technUr_Anbieterbindung})

Alternativ wäre eine Rechnung und Schätzung der Aufwände für eine Migration in der Theorie.
Diese Herangehensweise würde jedoch weitere Annahmen und Ungewissheiten mit sich bringen.
Daher wird argumentiert, dass die Aussagekraft dieser theoretischen Experimente nicht gegeben wäre.

\paragraph{Probleme mit der Einschätzung von Branchenexperten}
Es wird angenommen, dass ein befragter Experte entweder auf Basis von Erfahrungswerten oder Fachliteratur Bewertungen treffen würde. Problematisch ist hier, dass, 
wenn die Bewertung des Experten ohnehin auf diesem bekannten Fällen und Literatur basiert,
auch unmittelbar dieses Wissen zur Verifizierung der Ergebnisse herangezogen werden kann.
Erschwerend kommt hinzu, dass das Ergebnis des Lösungsansatzes konkrete Zahlenwerte sind.

Es wird angenommen, dass ein interviewter Experte keinen solchen Zahlenwert, sondern aus dem Stegreif vermutlich eine Rangordnung zur Anbieterbindung bei führenden Anbietern aufstellen könnte.

Auf Basis dieser Argumentationsführung, wird eine Experteneinschätzung abgelehnt.

\paragraph{Diskussion der Option der Recherche}
In Relation zu den anderen Verifizierungsansätzen steht die Recherche bisherige Erkenntnisse zur Anbieterbindung als praktikabelste Lösung dar, die zugleich wenige Annahmen und Ungenauigkeiten einführt.
Aus diesem Grund wird die Effektivität des Lösungsansatz durch die Recherche gemessen.
Konkret wird hierfür die Methodik einer anderen Untersuchung mit dem Lösungsansatz der Arbeit verglichen.

\paragraph{Herausforderungen bei der Recherche konkreter Migrationen}
Dabei ist jedoch zu beachten, dass auf die Neutralität des Autors der Literatur zu achten ist.
Erfolgreiche und reibungslose Migrationen werden häufig als Werbung auf den Produktseiten der Cloud-Computing-Anbieter veröffentlicht.

Darüber hinaus wird angenommen, dass die Normalisierung der Daten in der Literatur eine Herausforderung ist,
denn der Aufwand einer Migration kann durch unterschiedliche Metriken ausgedrückt werden.
So muss herausgearbeitet werden, wie sich beispielsweise die Dauer einer Migration in die Gesamtkosten umrechnen lassen, wenn eine Quelle nur ersteres liefert. 
Zuletzt annehmbar ist, dass über generell mehr Migrationen berichtet wird, die erfolgreich sind.
Auch Unternehmen, die aufgrund einer Anbieterbindung eine Migration aufschieben, berichten nicht über Migrationen.  Diese Voreingenommenheit zum Erfolg, also, dass die Realität gegebenenfalls nicht neutral durch die Recherche wiedergegeben wird, ist zu beachten.
Dieser Umstand kann mit dem \glq Überlebensirrtum\grq\ (englisch: Survivorship Bias) beschrieben werden.
\parencite{Survivorship2026}

Daher wird im Rahmen dieser Arbeit nur die Methodik einer anderen Untersuchung mit der Herangehensweise und dem Lösungsansatz verglichen.
\paragraph{Abschließende Einordnung der Verifizierbarkeit}
Alleine die Variablen innerhalb eines Unternehmens, wie die Verfügbarkeit von befähigtem Personal zur erfolgreichen Durchführung einer Migration, wie das konkrete Ziel und der Ursprung einer Migration, erschweren einen sauberen Vergleich.
Dazu kommen Unbekannte durch die Verifizierungsmethode, wie die Neutralität der Autoren von der Literatur.

Diese Punkte sind bei der Betrachtung der Verifizierbarkeit zu berücksichtigen.