\chapter{Einleitung}\label{ch:einleitung}

\section{Heranführung}

Die Heranführung dient im Kontext der Arbeit als theoretische Grundlage für nachfolgende Abwägungen, als Ort für Hintergrundinformationen und zur Angleichung des Themenverständnisses im Hinblick auf Begriffsbedeutungen. 

\subsection{Theoretische Hintergründe des Cloud-Computing}

\paragraph{Definitionen von Cloud Computing}
Es ist zusammen mit der künstlichen Intelligenz einer der wohl schillerndsten und meistverwendetsten Begriffe in der \Gls{IT}-Branche wie Thorsten Hennrich in seinem Fachbuch zu Cloud Computing nach der Datengrundschutzverordnung einleitend formulierte. \parencite{Hennrich2023}

Pragmatischer ist die ISO Norm, nach der Cloud Computing ein \begin{quote}
Paradigma [ist],um einen netzwerkbasierten Zugang auf ein skalierbares und elastisches Reservoir gemeinsam nutzbarer physischer oder virtueller Ressourcen nach dem Selbstbedienungsprinzip und bedarfsgerechter Administration zu ermöglichen.
\parencite{ISO22123}
\end{quote}

Ergänzend dazu steht der Artikel vom US-amerikanischen Technologieunternehmen Microsoft mit dem Titel \glq Was ist die Cloud?\grq. 
Der Webartikel klärt aus Sicht des Cloud-Computing-Anbieters auf und definiert die Cloud als \glqq[…] Online-Speicherplatz, in dem Personen und Unternehmen ihre Dateien und Anwendungen speichern, die von überall mit einer Internetverbindung zugänglich sind.\grqq\parencite{CloudDefMicro2025}
Des Weiteren bietet \glq die Cloud\grq , laut dem Anbieter, \glqq Dienste wie Rechenleistung, Datenbanken, Netzwerke und Softwareanwendungen.\grqq
\parencite{CloudDefMicro2025}

\paragraph{Folgerungen aus den Definitionen}
Aus diesen Perspektiven zu der Cloud-Computing-Technologie lassen sich wichtige Grundlagen wie den Zusammenhang zwischen Internet und Cloud und die inherente Flexibilität als Fundament der Technologie ableiten.

IT-Kompontenen wie der Speicher für Daten, die zu jeder Organisation und jedem Unternehmen gehören und traditionell am gleichen Standort beziehungsweise \glsfirst{on-Premise} aufzufinden waren, können ausgelagert werden. 
Je nach Umsetzungsart sind die IT-Komponenten von Cloud-Computing trotzdem auf dem Grundstück des Unternehmens, im gleichen Land oder an einem völlig anderen Ort auf der Erde.

Immer jedoch in Rechenzentren (englisch: Data Centers, auch Cluster), was spezielle Räume oder Gebäude sind, die aus einer Vielzahl einzelner Rechenmaschinen bestehen und enorme Maßstäbe annehmen können. Die Größe ist oft lediglich durch die finanziellen Grenzen des Betreibers und die Strom- und Internet-Infrastruktur eingeschränkt. Dadurch ist dieser Informationssektor (Quatärsektor der Wirtschaft) wie auch die (Ur-)Produktion von Standortfaktoren wie Grundstücks- und Energiepreisen, sowie -Verfügbarkeit (\zB Wasserkraft), abhängig.
\parencite{Tanenbaum2013} 
\parencite{Wirtschaftssektoren2026}

Genutzt werden die Komponten immer per Fernzugriff und -steuerung über das Internet. 

Diese Ausgestaltung alleine ermöglicht schon deshalb eine höhere Flexibilität, da Mitarbeitende des Unternehmens aus der Ferne auf die Komponenten zugreifen können. 

Darüberhinaus ist ein wichtiger Bestandteil des Konzeptes die Skalierung beziehungsweise Erweiterung der Ressourcen. 
Die Skalierung beinhaltet, dass zusätzliche Komponenten auf Wunsch des Cloud-Computing-Kunden zugeschaltet werden können. Diese Skalierung ist in die Produkte einbaut und geschieht über einen definierten Prozess.

\paragraph{Begriffsunterscheidung Cloud und Cloud-Computing}
Spannend ist, dass Cloud und Cloud-Computing oft synonym ver\-wendet werden. So auch teilsweise im Informationsmaterial von Microsoft,
wo das Konzept von Cloud-Computing beschrieben wird, 
aber oft als \glq die Cloud \grq abgekürzt wird.
Zur Verwirrung trägt erschwerend bei, 
dass Microsoft einen zweiten Artikel mit dem Titel \glqq Was ist Cloud-Computing?\grqq veröffentlicht hat, 
der eine sehr ähnliche Definition beinhaltet, aber die Cloud selbst als Bezeichnung für das Internet beschreibt. \parencite{CloudDefMicro20252}
Diese Arbeit bezieht sich auf Cloud-Computing nach den explizit genannten Definitionen. 

\paragraph{Geschichte und Verbreitung der Technologie}
Die Definitionen geben einen Einblick in das vermeintliches Potential der Technologie.

\subparagraph{Verbindungen zu Mainframe Computing}
Doch das Konzept von zentralisierter Com\-puter-Ressourcen in einem Rechenzentrum (englisch Data Center) ist laut Andrew Tanenbaums \glq Structured Computer Organisation\grq eine Art \glqq Mainframe Computing V2.0\grqq . \parencite{Tanenbaum2013}
Bei dieser Archiktur wurde auf einen Großrechner, der meist in einem speziellen Gebäude oder Raum einer Instituion aufzufinden war, über ein Intranet und ein leistungsschwachen Terminal(-Computer) zugegriffen. Über die Terminals werden große Rechenaufgaben angestoßen und die Ergebnisse später betrachtet.
Die Architektur war notwendig, da die benötigte Rechenleistung ein solch großes Volumen einnahm. \parencite{Tanenbaum2013}

\subparagraph{Miniaturisierung von Transistoren}
Da die Transistoren (fundamentaler Baustein für Computer) mit unvergleichlicher Geschwindigkeit miniaturisiert und immer kompakter auf Computer-Chips platziert werden konnten, schrupfte auch der Raumbedarf. Dadurch wurde die Mainframe-Archiktur zunächst obsolet. Die Entwicklung der Transistorengröße ist jedoch von physikalischen Größen begrenzt und dadurch ein ewiges Phänomen.
Zu den physikalischen Grenzen zählen Energie-Verlust (englisch: energy dissipation), Strom-Abfluss (englisch: current leakage) und die Größe von Silikon-Atomen
Im Fachbuch von Tanenbaum, wird erwähnt, dass die Grenzen der Transistor-Miniaturisierung Schätzungen zufolge 2023 erreicht werden könnte. \parencite{Tanenbaum2013}

\subparagraph{Wiedergeburt von zentralisiertem Computing}
Die rasante Entwicklung der Berechnungsmöglichkeiten erzeugt einen Teufelskreis (englisch: virtuous circle), der daraus besteht, dass die technologischen Entwicklungen auch die Anforderungen und Erwartungen an Software steigerte, was wiederrum verbesserte Rechenleistung forderte. \parencite{Tanenbaum2013}
Daher vertritt diese Arbeit die Auffassung, dass die beschriebenen Gegebenheiten die erneute räumliche und organisatorische Zentralisierung begünstigt haben.

\subparagraph{Aktuelle Verbreitung von Cloud-Computing}
Statistische Befragungen der letzten Jahrzehnte von Führungskräften deutscher und internationaler Unternehmen, machen deutlich, dass das Thema die Aufmerksamkeit der Führungsebenen erreicht hat und Cloud-Computing in die Unternehmen eingezogen ist.

Exemplarisch geht dies aus Abbildung \ref{fig:A_Nutz_CC} hervor.
Die Abbildung zeigt ein Balkendiagramm,
das die Nutzung von Cloud-Computing über die Jahre 2011 bis 2024 beschreibt.
Hierbei zeigt die y-Achse die Verteilung der Ant\-worten der Be\-fragten in Prozent und die x-Achse die dazugehörigen Jahres\-zahlen.
Ein Datenpunkt pro Jahr zeigt den Anteil der Unternehmen, die bereits Cloud Computing nutzen und ein weiterer Datenpunkt zeigt den Anteil der Unternehmen, die den Einsatz noch planen.
Laut der Visualisierung stieg im Zeitraum der Untersuchung,
welcher sich auf insgesamt 13 Jahre beläuft, die Nutzung von Cloud-Compting von 28\% auf 98\%. 
Der Anteil der Befragten, die angaben,
dass sie die Technologie nur planen, sank von 22\% auf 0\%. 
Für die Untersuchung wurden Geschäftsführer und IT-Führungskräfte aus 503 Unternehmen befragt. \parencite{StatistaCCNutzung2024}

Im Kontext dieser Arbeit gibt die Befragung einen ersten Einblick in die Nutzungstrends der Technologie an.
Der hohe Nutzungsanteil zum Ende der Datenreihe,
lässt darauf schließen, dass sich die Entscheidungstragenden heute nicht mehr fragen, ob sie diese Technologie einsetzen,
sondern welcher Anbieter wie genutzt wird.

\begin{figure}
\begin{center}
\includegraphics[width=0.75\linewidth]{figures/statistic_id177484_umfrage-zur-nutzung-von-cloud-computing-in-deutschen-unternehmen-bis-2024}
\end{center}
\label{fig:A_Nutz_CC}\caption{Nutzung von Cloud Computing}
\end{figure}

\paragraph{Funktionsweise und gängige Architekturen}
Es gibt viele verschiedene Variantenn und Konfigurationen des Cloud-Computing.
Wie auch bei eigenen Rechenzentren aus einer Vielzahl von Architekturen und Herangehensweisen gewählt werden kann,
so gibt es bei der Wahl der Cloud Liefermodelle und Produktbausteine, 
die nach den Bedürftnissen des Kunden kombiniert werden können.

\subparagraph{Service-Modell des Cloud-Computing}
Das Grundprinzip von Auslagerung von IT-Komponenten lässt sich in verschiedene Stufen unterteilen.
Die Abbildung \ref{fig:A-cloud-stack} zeigt eine eigene Übersicht über verschiedene Service-Modelle der Cloud-Computing-Architektur
basierend auf der ISO-Norm zu Cloud Computing. \parencite{ISO22123}.
Dabei ist farblich (rot) hervorgeheben, welche Komponenten jeweils vom Cloud-Computing-Anbieter verwaltet werden.
Die übrigen Komponenten werden durch den Kunden selbst verwaltet. \parencite{Cloud1012026}
Die Cloud-Computing-Modelle stehen im Vergleich zur On-Premise-Lösung, wo von Anwendung, über Runtimes, dme Betriebsystem bis hin zu den Servern und der Netzwerk-Infrastruktur alles vom potentiellen Cloud-Computing-Kunden selbst verwaltet wird.

\begin{figure}
\begin{center}
\includegraphics[width=0.75\linewidth]{figures/CloudStack}
\label{fig:A-cloud-stack}\caption{Übersicht über verschieden Verwaltungsformen der Cloud}
\end{center}
\end{figure}

\subparagraph{Public und Private Cloud}
Neben der Auslagerung von Bausteinen der IT-Infrastruktur, wird auch in Public und Private Cloud unterschieden.
Dabei ist besonders hervorzuheben, dass Informationen, die in die Public Cloud abgelegt werden, nicht automatisch öffentlich sind.
Stattdessen hat der Cloud-Kunde bei der Verwendung der Public Cloud den Zugriff auf einen öffentlichen Vorrat aus Rechenressourcen. \parencite{Cloud1012026}

Im Gegensatz hierzu wird der Umfang der Rechenressourcen bei der Private Cloud festgelegt und dann für den Kunden fest zugeordnet. Spätere Erweiterungen der Rechenressourcen sind hier trotzdem möglich.

\subparagraph{Verwendung mehrerer Cloud-Anbieter}
Neben der Verwendung eines einzigen Cloud-Anbieters für alle Bedürftnisse, gibt es auch Architekturen, die eine Zusammenarbeit mehrerer Anbieter und Cloud-Computing-Varianten vorsehen.

Dazu zählen die simultane Verwendung von Public und Private Cloud, sowie eigener Rechenzentren,
was als Hybride Cloud bezeichnet wird, wenn mindestens zwei Varianten gleichzeitig eingesetz werden. Auch verschiedene Service-Modelle durch Hybrid-Cloud-Architektur kombinierbar. \parencite{Cloud1012026}

Außerdem eine gängige Architektur ist die Multi-Cloud, 
bei der mehrere Public Cloud oder Private Cloud Anbieter gleichzeitig verwendet werden.

Bei Hybride Cloud und Multi-Cloud ist vorteilhaft, dass von den Allein\-stellungs\-merkmalen mehrerer Anbieter gleichzeitig profitiert werden kann
und die Vorzüge der jeweiligen Modelle kombiniert werden können.
Jedoch kommt die Verwendung solcher Ansätze auch mit höherem organisatorischen Aufwand, denn es müssen Geschäftsbeziehungen zu mehreren Anbietern gepflegt werden
und die Verknüpfung der verschiedenen Produkte kann zusätzliche Kosten und Aufwand mit sich bringen.
Außerdem kann der Verbindungskanal zwischen Produkten von verschieden Anbietern ein Flaschenhals für den Informationsaustausch und eine Sicherheits-Schwachstelle sein.

\paragraph{Versprechungen und Vorteile}
Zu den Vorzügen zählen unter anderem Kosteneinsparung, verbesserte Skalierbarkeit, Wiederherstellugsmöglicheiten, Datensicherheit und weitere Punkte,
die im Kapitel 1.4 namens "utopische Versprechungen des Cloud Computings" im Buch Cloud Governance aufgeführt werden. \parencite{Mezzio2023}

Mit der aufzeigten Popularität bei Unternehmen (vergleiche Abbildung \ref{fig:A_Nutz_CC}),lässt sich argumentieren, dass diese Versprechungen für Entscheidungstragende handfest sind.

\subparagraph{Kosteneinsparung}
Bei IT-Dienstleistern im GKV-Markt, die ihre Anwendungen selbst verwalten, fallen erfahrungsgemäß für Folgendes Kosten an:
\begin{itemize}
\item[-] Mitarbeitende für die Entwicklung, Betrieb und Wartung von Anwendungen
\item[-] Lizenzen für Kaufsoftware (auch für IDEs und Code-Verwaltungsplattformen)
\item[-] Zertifizierung (bei selbst hergestellter Software)
\item[-] externe zusätzliche Mitarbeiter bei großen Projekten
\item[-] externe Entwicklungs-Projekte
\item[-] Schulungen von Mitarbeitenden für neue Technologien und Anwendungen
\item[-] zusätzliche Mitarbeitende für (Personal-)Verwaltung, Buchhaltung und Einkauf
\item[-] zusätzliche Büroflächen
\end{itemize}
Soll darüberhinaus, auch die Infrastruktur selbst verwaltet werden, fallen zusätzlich noch Kosten für Folgendes an. 
\begin{itemize}
\item[-] Grundstücke, Gebäude und Räumlichkeiten
\item[-] Strom-, Wasser- und Wärmeversorgung
\item[-] Mitarbeitende für Installation, Betrieb und Wartung von Hardware
\item[-] Anschaffung, Ersatzteile und Entsorgung von IT-Komponenten
\item[-] Sonstiges wie beispielsweise Möbel
\item[-] Sicherheit wie Überwachungskameras, Sicherheitskräfte und Zertifizierungen
\end{itemize}
Besonders zu unterstreichen sind die initialen Investitionen, die in dieser Herangehensweise notwendig sind.

Durch diese eigene Darstellung der Kostensituation wird gezeigt, welche Kosten durch den Verzicht auf Cloud-Computing und die Zusammenarbeit mit SaaS-Anbietern entstehen können.

Des Weiteren wird ein Kontext für die Kosten von Cloud-Computing geschaffen.
Welche wiederrum bei der Integration von Cloud-Computing im Unternehmen teilweise oder vollständig anfallen.
Zu welchem Grad Kosten eigener Infrastruktur und Software wegfallen, hängt vom Service-Modell und der gewählten Architektur ab.

Grundsätzlich werden für die erfolgreiche Integration von Cloud-Computing Mitarbeitende mit neuen Kompetenzen benötigt. Zu diesen Kompetenzen zählt die Kontrolle, Regulierung und Überwachung der Dienstleistungsvereinbarungen (kurz: SLA) mit den Cloud-Computing-Anbietern und der Cloud-Computing-Nutzung durch das Unternehmen. \parencite{Mezzio2023}

Auch zu beachten ist, dass Schulungen für die neuen Bedienoberflächen der SaaS-Anbieter notwendig werden. Hat der gewählte Anbieter größtenteils eigene Software im Angebot, so gibt es entsprechend wenige Optionen für Schulungsanbieter auf dem Markt, was zu hohen Schulungskosten führen kann.

In der Summe lässt sich annehmen, dass finanzielles Risiko und große anfängliche Investitionen für Unternehmen wegfallen, und somit das Cloud-Computing eine geringere finanzielle Belastung sein kann. Jedoch ist ein vollständiger Kostenvergleich nicht Gegenstand dieser Arbeit, und soll hier nur ergänzend die Arbeit einleiten.
 
\subparagraph{Skalierbarkeit}
Bei herkömmlichen Architekturen mit Hardware und Software in eigner Verwaltung durch das Unternehmen oder den IT-Dienstleister ist die Herausforderung bezüglich der Skalierung erfahrungsgemäß der fehlende Vorrat freier Rechenressourcen, sowohl bei kurzfristiger Skalierung durch Lastspitzen, als auch bei langfristigen Skalierungen  beim Wachstum des Unternehmens.

Zur Verbesserung der Anschaulichkeit werden Beispiele der gesetzlichen Krankenkasse aufgeführt, was im Abschnitt \ref{vorstellung-aok} konkret vorgestellt wird.

Bei kurzfristiger Skalierung über den üblichen Arbeitstag, aber auch über das Arbeitsjahr gibt es Zeiten und Zeiträume, in denen mehr Kapazitäten benötigt werden. D
Beispielsweise gibt es in der AOK Bayern und AOK PLUS ein Mitarbeiterportal für die Zeitbuchung und das Stellen von verschiedenen Anträgen wie Urlaubsanträgen. 
Hierbei gibt es sicherlich Lastspitzen am Morgen und am Nachmittag. 
Ein weiteres Beispiel ist die Kommunikationssoftware, 
die bei unternehmensweiten Besprechungen zwei- oder viermal im Jahr deutlich stärker ausgelastet wird.
Gibt es freie Ressorucen, die bei alltäglicher Auslastung im Leerlauf sind, dann können diese bei Bedarf dazu geschalten werden. Das bedeutet für die Endnutzer kürzere Wartezeiten bei der Benutzung von Anwendungen.

Im Gegensatz hierzu gibt es auch langfristige Skalierungen, wenn ein Unternehmen wachsen möchte. 
Bei der Krankenkasse AOK Bayern oder AOK PLUS ist starkes Wachstum in einem kurzen Zeitraum eher untypisch, 
da sich die Versichertenzahl üblicherweise gleichmäßig und allmählich erhöht.
Dennoch können neue Bedürftnisse im Bezug auf Rechen- oder Speicherleistung entstehen, 
wenn sich gesetzliche Vorgaben ändern oder neue Technologien auf den Markt kommen.
Gibt es hier freie Ressourcen, können ohne Ausfall alte Komponten ausgetauscht werden oder ohne Austausch direkt mehr Ressourcen verwendet werden.

\subparagraph{Wiederherstellungsmöglichkeiten}
Bei technischen Fehl\-funktionen, Schä\-den durch Feu\-er oder Wasser, oder bei Verschlüsselung von Daten ist es wichtig eine Momentaufnahme der eigenen Systeme zu haben, 
um die verlorenen Informationen wiederherzustellen. 
Je aktueller die Momentaufnahme dabei ist, 
desto weniger Daten gehen durch den Vorfall und die Wiederherstellung verloren.
Je mehr verloren geht, desto höher der finanzielle Schaden, der dadurch verursacht wird.

\begin{figure}
\begin{center}
\includegraphics[width=0.75\linewidth]{figures/statistic_id417433_umfrage-zu-vorbeugenden-technischen-it-sicherheitsmassnahmen-in-unternehmen-2018}
\end{center}
\label{SecMaß-Nutzung}\caption{Vorbeugende technische IT-Sicherheitsmaßnahmen}
\end{figure}

Die Abbildung \ref{SecMaß-Nutzung} zeigt ein Balkendiagramm über das Thema IT-Sicherheitsmaßnahmen aus dem Jahr 2018. Neben anderen Maßnahmen wurde das Erstellen von Backups für Daten abgefragt und jedes der 503 Industrieunternehmen gab an, das es diese Maßnahme im Einsatz hatte.
Die Abbildung zeigt außerdem weniger weit verbreitete Maßnahmen wie Intrusion Detection System (IDS) oder die Verwendung von Penetrationstests. \parencite{StatistaSecMassnahmen2018}

Dennoch lässt sich schlussfolgern, dass die Relevanz von Backup- beziehungsweise Wiederherstellungsmöglichkeiten vollständig erkannt wurde. Doch auch die richtige und häufige Durchführung ist relevant.

Wie im Abschnitt \ref{produkt-übersicht-saas} beschrieben, zählt zu den gängigen Produkten vieler SaaS-Anbieter ein Produkt zur Verwaltung und automatischen oder manuellen Erstellung von Sicherungen.

Durch die Werbung für das Produkt, könnten sich mehr Kunden, die auf SaaS setzen, mit dem Thema auseinandersetzen.

\subparagraph{Datensicherheit}
Das Thema Wiederherstellungsmöglichkeiten liefert einen Einstieg in die Art und Weise wie Saas-Anbieter Kunden zum Oberthema Datensicherheit heranführen können.
Durch das Angebot von diversen Produkten werden die Themen auch für Entscheidungstragende greifbarer, der Einstieg wird leichter und der Bedarf nach Fachpersonal sinkt.

Letzeres ist ein Punkt der ein großes Gewicht hat, denn Fachpersonal ist oftmals eine Hürde bei der Digitalisierung allgemein und damit auch speziell bei Sicherheitsmaßnahmen \parencite{StatistaHuerden2025}

Wird die Thematik in ein oder mehrere leicht bedienbare Produkte verpackt, für deren Bedienung nur wenige Mitarbeitende nötig sind, so steigt die Sicherheit durch die Verwendung von SaaS-Produkten.

\subparagraph{Weitere Punkte}
Außerdem werden im ursprünglich zitierten Kaptiel noch \glqq Vorteile der Cloud\grqq\ wie \glq Internet of Things\grq -Funktionalitäten,
ver\-besserte Zu\-sammen\-arbeit, Umwelt\-freund\-lich\-keit,
Mitarbeiter-Engagement, Echtzeit-Software-Updates und Analyse-Möglich\-keiten angesprochen. \parencite{Mezzio2023}

Da diese Punkte jedoch in der Gesamtaufstellung auf den hinteren Plätzen der Einordnung der Quelle landen, werden diese in diesem gemeinsamen Abschnitt kurz aufgeführt.

Besonders hervorzuheben ist der letzte Platz der Auflistung,
der wie auch bei anderen Vorteilen bereits, 
durch Produkte bei vielen Anbietern ermöglicht wird. 
Die Analyse von Daten, 
Netwerkverkehr oder sonstigen Informationen wird durch eigene Produkte für die Kunden von SaaS-Anbietern stark vereinfacht und zugänglicher gemacht.

\paragraph{Kapitalisierung von Cloud-Computing}
Aus dem Paradigma Cloud Computing und den verschiedenen aufgezeigten Service-Modellen (vergleiche Abschnitt \ref{fig:A-cloud-stack}), ergibt sich ein zentrales Geschäftsmodell.

\subparagraph{Computing als Dienstleistung}
Statt Produkte wie beispielsweise Serverkomponenten zu kaufen, wird Computing zu einer Dienstleistung. Dabei werden einmalige Anschaffungskosten und Betriebskosten zu einer monatlichen oder jährlichen Gebühr.
Durch die neue Beziehung zwischen Cloud-Computing-Anbieter und Kunde ergeben sich Anforderungen für eine geregelte Geschäftsbeziehung. \parencite{Cloud1012026}
Die notwendigen Regeln werden in den Dienstleistungsvereinbarungen (englisch: Service-Level-Agreement, SLA) in einvernehmlich und auf Basis geltender rechtlicher Vorschriften geschlossen.
Die Vereinbarungen regeln die genaue Beschaffenheit und Eigenschaften jeder Leistung des Anbieters und die Ansprüche des Kunden im Falle eines Verstoßes. Unter Anderem zudem vertraglich geregelt, sind Punkte wie Bepreisung der Leistungen und der zeitliche Rahmen der Vereinbarung. 

%Wie im Abschnitt 
%
%Der Bedarf nach den Leistungs\-versprechungen der Cloud ist enorm.
%Nicht nur das Trainieren, sondern auch das Nutzen von KI-Modellen auf der eigenen Hardware ist rechenaufwändig. Auch andere Anwendungen wie das Managment von Kunden und Unternehmensressourcen wird immer komplexer.
%
%Grundlegend kann gewählt werden wie viele Schichten des ursprünglichen Rechenzentrums in die Cloud gehoben werden sollen. (vergleiche \ref{fig:A-cloud-stack})

\paragraph{Schematische Gegenüberstellung von Kosten und Funktionsweise}

\paragraph{Herausforderungen und Nachteile}
Die Herausvorderungen und Nachteile der Cloud werden im späteren Kapitel 3.6 "Der organisatorische Einfluss von Cloud-Computing" des Buches "Cloud Governance" aufgelistet:
\begin{itemize}
\item[-] Sicherheit (gegen Cyber-Angriffe)
\item[-] Kosten(-regulierung)
\item[-] (Integration von) Alt-Anwendungen
\item[-] Ausfälle
\item[-] Anbieterbindung
\item[-] (Verlust von) technischem Fachwissen
\end{itemize}
Die Auf\-zählung wurde aus dem Eng\-lischen über\-setzt und es wurde Kon\-text er\-gänzt.
\parencite{Mezzio2023}
Die konkreten Punkte stammen aus einem Blog-Artikel der IT-Sicherheits\-firma Conosco.
\parencite{Conosco2020}
Der Fokus dieser Arbeit ist die Anbieterbindung.
Eine tatsächliche Anbieterbindung ist nur auf der Ebene Software as a Service möglich.

\subsection{Grundlagen der Anbieterbindung}

\paragraph{Definitionen von Anbieterbindung}
Vendor-Lock-In (dt. Anbieterbindung) ist ein Umstand in der Beziehung zu einem Anbieter aus Kundensicht. 
Dieser Umstand wird beim Beenden der Beziehung problematisch.
Wenn möchte ein Unternehmen den Anbieter wechseln,
so ist eine Anbietermigration nötig.

Grundsätzlich ist die Migration von einem Anbieter zu einem Konkurrenten immer mit Aufwand verbunden,
wenn beispielsweise die Daten einer CRM-Anwendung des einen Anbieters zum Anderen gesendet werden müssen.
Problematisch wird es dann,
wenn die Daten in unterschiedlichen Formaten abgespeichert sind. 
Hilfreich sind dann Werkzeuge zur Migration,
welche bei branchenüblichen CRM-Anwendungen leicht erhältlich sind. 
Waren beim ursprünglichen Anbieter jedoch Anbieter-eigene Lösungen im Einsatz,
steigert sich der Migrationsaufwand über das zu erwartende Pensum hinaus.

Ein besonders schwerwiegender Vendor-Lock-In liegt vor, wenn Bausteine des Produkts gar nicht vom neuen Anbieter abgebildet werden können.

\paragraph{Technische Ursache für Anbieterbindung}
Die Abbildung \ref{fig:technUr_Anbieterbindung} skizziert, wie das Paradigma Cloud-Computing und im Detail das SaaS-Service-Modell Anbieterbindung hervorbringt.
Es werden Abhängigkeiten in der CC-Umgebung schematisch dargestellt,um zu visualisieren welche Verbindungen bei einem Anbieterwechsel neu geknüpft werden müssten.

In der Abbildung befinden sich die vier Kernelemente Kunde (1), Anbieter (2), Rechenressourcen (3) und Daten (ohne Nummerierung). Im Zentrum steht dabei der Anbieter, vertreten durch seine Programme.

\begin{figure}
\begin{center}
\includegraphics[width=0.75\linewidth]{figures/technUr_Anbieterbindung}
\end{center}
\label{technUr_Anbieterbindung}\caption{Eigene Skizze der Beziehung zwischen Cloud-Computing-Kunde, CC-Anbieter und den Rechenressourcen}
\end{figure}




\paragraph{Anfällige Cloud-Produkte}
In niedrigeren Liefermodellen wie Platform as a Service,
wo die Anwendungen und Daten in der Hand des Kunden liegen, ist Anbieterbindung generell kein Problem.
Dadurch,
dass die Anwendungen bei allen Anbietern gleichermaßen durch den Kunden gewählt und betrieben werden,
können diese Anwendungen zum neuen Anbieter einfach migriert werden.

Dies ist ebenfalls aufwendig, vor allem wenn die Umgebung beim neuen Anbieter anders ist. Jedoch ist eine Migration pauschal immer möglich.

Die Kunden habe dazu das technische Fachwissen im Hause,
um die Anwendungen entsprechen zu modifzieren oder umzukonfigurieren,
damit diese in der neuen Umgebung funktionieren.

\paragraph{Einordnung der Kritikalität des Problems}


\paragraph{Erkennung von betroffenen Cloud-Produkten}

\paragraph{Werkzeuge zur Messung von Anbieterbindung}
Zudem lassen sich Vorteile wie der Kostenpunkt leicht quantifizieren und damit vergleichen.
Auch die Skalierbarkeit lässt sich durch die Zeit messen, die es benötigt zusätzliche Hardware einzubinden, wenn beispielsweise hoher Verkehr es fordert.
Ferner lassen sich auch andere Eigenschaften der verschieden Cloud-Produkte wie die Anzahl der Backups oder die Anzahl von Datenlecks gegenüberstellen.


\subsection{Überblick über führende Cloud-Computing-Anbieter}
\label{lead-cc-A}
\subsection{Überblick über Cloud-Produkte}
\label{produkt-übersicht-saas}

\paragraph{Kernelemente der Produktkataloge von Anbietern}

\paragraph{Navigieren von Produktkatalogen}

\section{Motiviation}

Die Motivation unterstreicht im Kontext der Arbeit die Notwendigkeit zur Erfüllung der Ziele und umreißt kurz die Gründe für die Themenwahl.

\subsection{Bedarf eines Prozesses zur Einordnung von Anbieterbindung}

\paragraph{Strategische Bedeutung für Unternehmenen}
Die Beschäftigung mit Anbietern ist spannend, denn sie hat strategische und politische Komponenten.
Die Wahl eines Cloud-Anbieters für ein Unternehmen ist elementar und Anbieterbeziehungen durchlaufen einen Lebenszyklus (vergleiche Software-Lebenszyklus).
Obwohl es Diskrepanzen zwischen der Praxis unter der Theorie gibt, so sollte schon bei der Schließung einer neuen Geschäftsbeziehung deren Ende und Wechsel-Strategie festgelegt sein.
Hierfür ist die Durchleuchtung eines Anbieters hinsichtlich Vendor-Lock-In schon im Voraus wichtig.
Wie schon festgestellt, wird die Anbieterbindung beim Beenden einer Geschäftsbeziehung relevant.
Dafür zentral ist, wann das Ende der Geschäftsbeziehung in einem Unternehmen erreicht ist.
Bei alleinstehenden Anwendungen beispielsweise wird die Lebenszeit überlicherweise auf eine gewisse Jahreszahl begrenzt. 
Allerdings können wie auch bei den klassischen  Anwendungen bei einem Cloud-Anbieter Bedingungen eintreffen, die einen früheren Wechsel verlangen.

Diese Bedingungen können finanzieller Art sein. So könnte etwa der aktuelle Anbieter in Anbetracht seiner Leistungen nicht mehr wirtschaftlich sein.

Nicht nur finanzielle Aspekte können zu einem Wechselwunsch beim Kunden führen.

Durch Anpassungen am Leistungskatalog und die vertragliche Möglichkeit manche Leistungen nicht mehr anzubieten, kann es dazu kommen, dass notwenige Bausteine nicht mehr vom Cloud-Computing-Anbieter unterstützt werden.
Solche Anpassungen sind aufgrund der festen Vertragsregeln zwar nie plötzlich, meistens aber ein Argument für einen Wechsel.

Außerdem kann es dazu kommen, 
dass Kunden von mehreren Anbietern ihre benötigten Leistungen auf einen einzigen konsolidieren wollen oder im Gegenbeispiel ihre Anforderungen auf mehrere Anbieter verteilen wollen,
um die unterschiedlichen Alleinstellungsmerkmale mehrerer Anbieter gleichzeitig zu nutzen.

Zuletzt kann es auch durch äußere Faktoren wie gesetzliche Vorgaben,
denen das Produkt des aktuellen Anbieters nicht mehr folgt, dazu kommen,
dass ein Wechsel unbedingt notwendig wird.
Auch geopolitische Änderungen wie Zölle oder Gesetze zählen zu den Gründen für das frühzeitige Ende der Geschäftsbeziehung.

\paragraph{Strategische Bedeutung auf geopolitscher Ebene}

\subsection{Besonderheiten des Themas}

\paragraph{Unterschiede zu anderen Herausforderungen der Cloud}
Da darüberhinaus die Anbieterbindung nicht während der Lebenszeit einer Anbieterbeziehung sondern am Ende ins Gewicht fällt, ist die Auseinandersetztung nicht so allgegenwärtig wie andere Cloud-Themen.

\subsection{Beenden von Anbieterbeziehungen}

\section{Abgrenzungen}
In der Abgrenzung werden im Kontext dieser Arbeit gezielt und explizit Inhalte ausgeschlossen, um die Zielsetzung möglichst geradlinig und anwendungsbezogen zu erfüllen.
\subsection{Distanzierung von ökonomischen Ansätzen}
Im Gegensatz dazu sind vertragliche oder ökonomische Kriterien Gegenstand dieser Arbeit.
Zur Verdeutlichung wird also beispielsweise nicht untersucht, ob die These, dass das Nutzen eines teurerer Cloud-Computing-Anbieter seltener zum Vendor-Lockin führt, zutrifft.

\subsection{Fokus auf öffentliche Cloud}

\subsection{Eingrenzung der zu analysierenden Anbieter}

\paragraph{Kennzahlen für Cloud-Anbieter}

\paragraph{Einblick in die Marktanteile in Deutschland}
Bei der Auswahl der Anbieter wurden sowohl solche berücksichtigt, die die kubus IT bereits verwendete, als auch solche die vermeintlich interessante Ergebnisse liefern sollten.
Aktuell sind folgende Cloud-Computing-Anbieter bereits in Benutzung.
\begin{itemize}
\item[-] Arvato
\item[-] Microsoft Azure
\end{itemize}
Darüberhinaus werden folgende Anbieter aufgrund ihrer Relevanz auf dem internationalen Markt, ihrer besonderen Größe oder ihrer Relevanz für deutsche Firmen berücksichtigt.
\begin{itemize}
\item[-] Amazon Warehoue Services (Vereinigte Staaten)
\item[-] Google Cloud Plattform (Vereinigte Staaten)
\item[-] Alibaba Cloud (China)
\item[-] IONOS Cloud (Deutschland)
\end{itemize}


