\chapter{Einleitung}\label{ch:einleitung}

\section{Heranführung}

Die Heranführung dient im Kontext der Arbeit als theoretische Grundlage für nachfolgende Abwägungen, als Ort für Hintergrundinformationen und zur Angleichung des Themenverständnisses im Hinblick auf Begriffsbedeutungen. 

\subsection{Theoretische Hintergründe des Cloud-Computing}
\paragraph{Cloud Computing als Phänomen}
Es ist zusammen mit der künstlichen Intelligenz einer der wohl schillerndsten und meistverwendetsten Begriffe in der \Gls{IT}-Branche wie Thorsten Hennrich in seinem Fachbuch zu Cloud Computing nach der Datengrundschutzverordnung einleitend formulierte.
Der Technologie Cloud Computing wird das \glqq disruptive Potential\grqq\ zugeorndet, das den Startschuss einer \glqq neuen Ära in der Informationstechnologie\grqq bedeuten könnte. \parencite[Kap.~1.2,S.~17]{Hennrich2023}

\paragraph{Definitionen von Cloud Computing}
Laut der ISO Norm, ist Cloud Computing ein \begin{quote}
Paradigma,um einen netzwerkbasierten Zugang auf ein skalierbares und elastisches Reservoir gemeinsam nutzbarer physischer oder virtueller Ressourcen nach dem Selbstbedienungsprinzip und bedarfsgerechter Administration zu ermöglichen.
\parencite{ISO22123}
\end{quote}

ISO ist die Kurzbezeichnung der internationalen Organisation für Standardisierung. Die drei Buchstaben sind laut dem Internetauftritt der Organisation vom griechischen \glq isos \grq abgeleitet, was zu \glq gleich \grq bedeutet.
Die Standards der Organisation werden durch Komitees entwickelt. \parencite{ISODef2026}

Ergänzend zu dieser Definition steht der Artikel vom US-amerikanischen Technologieunternehmen Microsoft mit dem Titel \glq Was ist die Cloud?\grq. 
Der Webartikel klärt aus Sicht des Cloud-Computing-Anbieters auf und definiert die Cloud als \glqq[…] Online-Speicherplatz, in dem Personen und Unternehmen ihre Dateien und Anwendungen speichern, die von überall mit einer Internetverbindung zugänglich sind.\grqq\parencite{CloudDefMicro2025}
Des Weiteren bietet \glq die Cloud\grq , laut dem Anbieter, \glqq Dienste wie Rechenleistung, Datenbanken, Netzwerke und Softwareanwendungen.\grqq
\parencite{CloudDefMicro2025}

Dabei ist anzumerken, dass jeder Anbieter für Cloud-Computing eine Variation dieser Definition veröffentlich. Diese Arbeit beschränkt sich schwerpunktmäßig auf die ISO-Definition. 

\paragraph{Folgerungen aus den Definitionen}
Aus diesen Perspektiven zu der Cloud-Computing-Technologie lassen sich wichtige Grundlagen, wie den Zusammenhang zwischen Internet und Cloud-Computing und die inherente Flexibilität als Fundament der Technologie, ableiten.

\subparagraph{Ressource Internet im Kontext der Technologie}
IT-Komponenten, wie der Speicher für Daten, die zu jeder Organisation und jedem Unternehmen gehören und traditionell am gleichen Standort beziehungsweise \glsfirst{on-Premise} aufzufinden waren, können ausgelagert werden. 
Je nach Umsetzungsart sind die IT-Komponenten von Cloud-Computing trotzdem auf dem Grundstück des Unternehmens, im gleichen Land oder an einem völlig anderen Ort auf der Erde. Jede Komponente, die der Kunde nicht selbst verwaltet, wird durch eine hinreichende Abstraktionsebene versteckt und es wird nur eine \glq bedarfsgerechte Administration ermöglicht\grq (vergleiche ISO-Definition).

Immer jedoch, sind die Komponenten in Rechenzentren (englisch: Data Centers, auch Cluster), 
was spezielle Räume oder Gebäude sind, die aus einer Vielzahl einzelner Rechenmaschinen bestehen und enorme Maßstäbe annehmen können. 
Die Größe ist oft lediglich durch die finanziellen Grenzen des Betreibers und die Strom- und Internet-Infrastruktur eingeschränkt.
Es wird eine Internetverbindung benötigt, 
die einzelne Daten schnell und viele Daten parallel transportieren kann. 
Analog ist eine stabile und leistungsfähige Stromanbindung für Rechenzentren nötig. 
Stromausfälle beispielsweise schränken die Verfügbarkeit ein und können die Komponenten beschädigen.
Dadurch ist Cloud-Computing als Produkt des Informationssektor (Quatärsektor der Wirtschaftsektoren) 
wie auch die (Ur-)Produktion von Standortfaktoren wie vorteilhaften Grundstücks- und Energiepreisen,
sowie -Verfügbarkeit (\zB Wasserkraft), abhängig.
\parencite[Kap.~1.3]{Tanenbaum2013} 
\parencite{Wirtschaftssektoren2026}

\subparagraph{Flexibilität per Design}
Genutzt werden die Komponten folglich per Fernzugriff und -steuerung über das Internet. (vergleiche ISO-Definition)

Diese Ausgestaltung alleine ermöglicht schon deshalb eine höhere Flexibilität, da Mitarbeitende des Unternehmens aus der Ferne auf die Komponenten zugreifen können. 

Darüberhinaus ist ein wichtiger Bestandteil des Konzeptes die Skalierung beziehungsweise Erweiterung der Ressourcen. 
Die Skalierung beinhaltet, 
dass zusätzliche Komponenten auf Wunsch des Cloud-Computing-Kunden zugeschaltet werden können. 
Diese Skalierung ist in die Produkte eingebaut und geschieht \glqq nach Bedarf\grqq über einen definierten Prozess.
Unabhängig von diesem Prozess entfällt die Notwendigkeit auf Kundenseite zusätzliche physische Komponenten an das System anzuschließen.
\parencite[Kap.~1.2,S.~17]{Hennrich2023}

Abrechnungsmodelle in dieser Flexibilität werden übersichthalber in Abschnitt \ref{abrechnungsmodelle} kurz vorgestellt.

\paragraph{Begriffsunterscheidung Cloud und Cloud-Computing}
Cloud und Cloud-Computing wird oft synonym ver\-wendet werden. 
So auch teilsweise im Informationsmaterial von Microsoft,
wo das Konzept von Cloud-Computing beschrieben wird, 
aber oft als \glq die Cloud \grq\ abgekürzt wird.
Zur Irritation trägt erschwerend bei, 
dass Microsoft einen zweiten Artikel mit dem Titel \glqq Was ist Cloud-Computing?\grqq\ veröffentlicht hat, 
der eine sehr ähnliche Definition beinhaltet, aber die Cloud selbst als Bezeichnung für das Internet beschreibt. \parencite{CloudDefMicro20252}

Diese Arbeit bezieht sich auf Cloud-Computing nach den explizit genannten Definitionen und, wenn alternative Begriffe synonym zu verstehen sind, wird dies ausdrücklich gekennzeichnet.

\paragraph{Geschichte und Verbreitung der Technologie}
Die Definitionen geben einen Einblick in das vermeintliche Potential der Technologie.
Die tatsächliche Popularität von zentralisierter Computer-Ressorucen und Cloud-Computing hat sich im Laufe der Zeit verändert.

\subparagraph{Verbindungen zu Mainframe Computing}
Das Konzept Cloud-Computing,
also von zentralisierter Com\-puter-Ressourcen in einem Rechenzentrum (englisch Data Center),
beschreibt A. Tanenbaum als eine Art \glqq Mainframe Computing V2.0\grqq .
\parencite[Kap.~1.3]{Tanenbaum2013}
Bei dieser Archiktur wurde auf eine Rechenmaschiene,
die aufgrund des Platz- und Strombedarfs meist in einem speziellen Gebäude oder Raum einer Instituion aufzufinden war,
über ein Intranet per Terminal(-Computer) zugegriffen.
Über diese Terminals werden große Rechenaufgaben angestoßen und die Ergebnisse später betrachtet.
Sie waren die Benutzeroberfläche. \parencite[Kap.~1.3]{Tanenbaum2013}

\subparagraph{Miniaturisierung von Transistoren}
Da die Transistoren (fundamentaler Baustein für Computer) mit unvergleichlicher Geschwindigkeit miniaturisiert und immer kompakter auf Computer-Chips platziert werden konnten, schrupfte auch der Raumbedarf.
Dadurch wurde die Mainframe-Archiktur zunächst obsolet.
Die Entwicklung der Transistorengröße ist jedoch von physikalischen Größen begrenzt und dadurch ein ewiges Phänomen.
Zu den physikalischen Grenzen zählen Energie-Verlust (englisch: energy dissipation), Strom-Abfluss (englisch: current leakage) und die Größe von Silikon-Atomen.
Im Fachbuch von Tanenbaum, wird erwähnt, dass die Grenzen der Transistor-Miniaturisierung Schätzungen zufolge 2023 erreicht werden könnte. \parencite[Kap.~1.3]{Tanenbaum2013}

\subparagraph{Wiedergeburt von zentralisiertem Computing}
Die rasante Entwicklung der Berechnungsmöglichkeiten erzeugt einen Teufelskreis (englisch: virtuous circle), der daraus bestand,
dass die technologischen Entwicklungen auch die Anforderungen und Erwartungen an Software steigerten,
was wiederrum verbesserte Rechenleistung forderte. \parencite{Tanenbaum2013}

Daher vertritt diese Arbeit die Auffassung, dass die beschriebenen Gegebenheiten die erneute räumliche und organisatorische Zentralisierung begünstigt haben.

\subparagraph{Aktuelle Verbreitung von Cloud-Computing}
Statistische Befragungen der letzten Jahrzehnte von Führungskräften deutscher und internationaler Unternehmen, machen deutlich, dass das Thema die Aufmerksamkeit der Führungsebenen erreicht hat und Cloud-Computing in die Unternehmen eingezogen ist.

Exemplarisch geht dies aus Abbildung \ref{fig:A_Nutz_CC} hervor.
Die Abbildung zeigt ein Balkendiagramm,
das die Nutzung von Cloud-Computing über die Jahre 2011 bis 2024 beschreibt.
Hierbei zeigt die y-Achse die Verteilung der Ant\-worten der Be\-fragten in Prozent und die x-Achse die dazugehörigen Jahres\-zahlen.
Ein Datenpunkt pro Jahr zeigt den Anteil der Unternehmen, 
die bereits Cloud Computing nutzen und ein weiterer Datenpunkt zeigt den Anteil der Unternehmen, die den Einsatz noch planen.
Laut der Visualisierung stieg im Zeitraum der Untersuchung,
welcher sich auf insgesamt 13 Jahre beläuft, die Nutzung von Cloud-Compting von 28\% auf 98\%. 
Der Anteil der Befragten, die angaben,
dass sie die Technologie nur planen, sank von 22\% auf 0\%. 
Für die Untersuchung wurden Geschäftsführer und IT-Führungskräfte aus 503 Unternehmen befragt. \parencite{StatistaCCNutzung2024}

Im Kontext dieser Arbeit gibt die Befragung einen ersten Einblick in die Nutzungstrends der Technologie an.
Der hohe Nutzungsanteil zum Ende der Datenreihe,
lässt darauf schließen, dass sich die Entscheidungstragenden heute nicht mehr fragen, ob sie diese Technologie einsetzen,
sondern welcher Anbieter wie genutzt wird.

\begin{figure}
\begin{center}
\includegraphics[width=0.75\linewidth]{figures/statistic_id177484_umfrage-zur-nutzung-von-cloud-computing-in-deutschen-unternehmen-bis-2024}
\end{center}
\label{fig:A_Nutz_CC}\caption{Nutzung von Cloud Computing}
\end{figure}

\subsection{Aktuelle Implementierungen des Paradigmas}
\paragraph{Kapitalisierung von Cloud-Computing}
Aus dem Paradigma Cloud Computing und den verschiedenen aufgezeigten Service-Modellen (vergleiche Abbildung \ref{fig:A-cloud-stack}), ergibt sich ein zentrales Geschäftsmodell.

\subparagraph{Computing als Dienstleistung}
Statt Produkte wie beispielsweise Serverkomponenten zu kaufen, wird Computing zum Produkt in Form einer Dienstleistung. 
Dabei werden einmalige Anschaffungskosten und Betriebskosten zu einer monatlichen oder jährlichen Gebühr.
\parencite[Kap.~1.2,S.~17]{Hennrich2023}
Durch die neue kontinuierliche Beziehung zwischen Cloud-Computing-Anbieter und Kunde ergeben sich Anforderungen für eine geregelte Geschäftsbeziehung,
zur klaren Regelung aller Bestandteile der fortlaufenden Interaktion. 
\parencite[Kap.~8, S.~89ff]{Mezzio2023}

Dieses Regelwerk wird als Dienstleistungsvereinbarung (englisch: Service-Level-Agreement, SLA)
bezeichnet und von der europäischen Kommision als
\begin{quote}
Das Instrument zur Steuerung der Beziehung zwischen dem Endnutzer (Cloud Service Customer) und dem Dienstleister (Cloud Service Provider),
ist ein einvernehmlicher Vertrag zwischen den beiden Parteien,
was als Cloud Service Level Agreement (CSLA) bezeichnet wird. Aus der globalen Natur von Cloud-Angeboten heraus, übertreten CSLAs typischerweise mehrere Rechtsräume.
Diese unterschiedliche Handhabung ist besonders im Hinblick auf Schutz von personenbezogenen Daten im Cloud-Service relevant.
Die Vereinbarungen unterscheiden sich bei jedem CSP insofern,
dass die Grundfunktionalitäten Ähnlichkeiten beinhalten,
aber die individuellen Regelungen und Bedingungen für die Dienste der jeweiligen Anbieter sind einzigart für diesen Anbieter.
\parencite[Übersetzt für diese Arbeit]{SLADefEU2016}
\end{quote}
definiert. 

\subparagraph{Folgerungen aus der Definition}
Aus der Definition der europäischen Kommision lässt sich folgern, dass eine korrekte Ausgestaltung der CSLAs entscheidend für positive Zusammenarbeit aus Sicht Cloud Service Customers ist.
Die Aufgabe des Kunden ist es im Detail zu untersuchen, ob die Vereinbarung, die ein Cloud Service Provider vorschlägt oder in Abstimmung erarbeitet hat, tatsächlich vorteilhaft für die künftige Zusammenarbeit und vor allem hilfreich für die Klärung von Uneinigkeiten ist.
Ist dies nicht der Fall, so könnnen unvorteilhafte oder unklare Formulierungen zu Rechtsstreitigkeiten und einer Gefahr für die langfristige Geschäftsstrategie des Kunden werden.
\parencite{SLADefEU2016},\parencite[S.~89ff]{Mezzio2023}

%Wie im Abschnitt 
%
%Der Bedarf nach den Leistungs\-versprechungen der Cloud ist enorm.
%Nicht nur das Trainieren, sondern auch das Nutzen von KI-Modellen auf der eigenen Hardware ist rechenaufwändig. Auch andere Anwendungen wie das Managment von Kunden und Unternehmensressourcen wird immer komplexer.
%
%Grundlegend kann gewählt werden wie viele Schichten des ursprünglichen Rechenzentrums in die Cloud gehoben werden sollen. (vergleiche \ref{fig:A-cloud-stack})

\paragraph{Schematische Gegenüberstellung von Abrechnungsmodell und Funktionsweise}
\label{abrechnungsmodelle}
Grundlegend sind die Abbrechnungsmodell der Cloud-Computing-Anbieter so gestaltet, 
dass die Nutzung der Kunden einberechnet wird.
Obwohl es vereinzelt einmalige Kosten wie Installations- oder Migrationsgebühren gibt,
richten sich die Kosten für den Kunden prinzipiell nach der Anzahl der genutzen Ressourcen oder der Dauer der Nutzung.
\begin{tabular}{l l}
Bezeichnung & Prinzip \\
\glq as a Service\grq & Produkt mit Abonnement \\
\glq on Demand\grq & Produktkatalog mit Abonnement \\
\glq Pay per Use\grq & Festgelegter Preis / Einheit \\
\end{tabular}
\\
\parencite[Kap.~1.2]{Hennrich2023}

\paragraph{Funktionsweise und gängige Architekturen}
Es gibt viele verschiedene Varianten und Konfigurationen des Cloud-Computing.
Wie auch bei eigenen Rechenzentren aus einer Vielzahl von Architekturen und Herangehensweisen gewählt werden kann,
so gibt es bei der Wahl der Cloud Liefermodelle und Produktbausteine, 
die nach den Bedürfnissen des Kunden kombiniert werden können.

\subparagraph{Service-Modell des Cloud-Computing}
Das Grundprinzip von Auslagerung von IT-Komponenten lässt sich in verschiedene Stufen unterteilen.
Die Abbildung \ref{fig:A-cloud-stack} zeigt eine eigene Übersicht über verschiedene Service-Modelle der Cloud-Computing-Architektur
basierend auf der ISO-Norm zu Cloud Computing. \parencite{ISO22123}.
Dabei ist farblich (rot) hervorgeheben, welche Komponenten jeweils vom Cloud-Computing-Anbieter verwaltet werden.
Die übrigen Komponenten werden durch den Kunden selbst verwaltet. \parencite{Cloud1012026}
Die Cloud-Computing-Modelle stehen im Vergleich zur On-Premise-Lösung, wo von Anwendung, über Runtimes, dme Betriebsystem bis hin zu den Servern und der Netzwerk-Infrastruktur alles vom potentiellen Cloud-Computing-Kunden selbst verwaltet wird.

\begin{figure}
\begin{center}
\includegraphics[width=0.75\linewidth]{figures/CloudStack}
\label{fig:A-cloud-stack}\caption{Übersicht über verschieden Verwaltungsformen der Cloud}
\end{center}
\end{figure}

\subparagraph{Public und Private Cloud}
Neben der Auslagerung von Bausteinen der IT-Infrastruktur, wird auch in Public und Private Cloud unterschieden.
Dabei ist besonders hervorzuheben, dass Informationen, die in die Public Cloud abgelegt werden, nicht öffentlich zugänglich sind.
Stattdessen hat der Cloud-Kunde bei der Verwendung der Public Cloud den Zugriff auf einen öffentlichen Vorrat aus Rechenressourcen. \parencite{Cloud1012026}

Im Gegensatz hierzu wird der Umfang der Rechenressourcen bei der Private Cloud festgelegt und dann für den Kunden fest zugeordnet. Spätere Erweiterungen der Rechenressourcen sind hier trotzdem möglich.

\subparagraph{Verwendung mehrerer Cloud-Anbieter}
Neben der Verwendung eines einzigen Cloud-Anbieters für alle Bedürfnisse, gibt es auch Architekturen, die eine Zusammenarbeit mehrerer Anbieter und Cloud-Computing-Varianten vorsehen.

Dazu zählen die simultane Verwendung von Public und Private Cloud, sowie eigener Rechenzentren,
was als Hybride Cloud bezeichnet wird, wenn mindestens zwei Varianten gleichzeitig eingesetz werden. Auch verschiedene Service-Modelle durch Hybrid-Cloud-Architektur kombinierbar. \parencite{Cloud1012026}

Außerdem eine gängige Architektur ist die Multi-Cloud, 
bei der mehrere Public Cloud oder Private Cloud Anbieter gleichzeitig verwendet werden.

Bei Hybride Cloud und Multi-Cloud ist vorteilhaft, dass von den Allein\-stellungs\-merkmalen mehrerer Anbieter gleichzeitig profitiert werden kann
und die Vorzüge der jeweiligen Modelle kombiniert werden können.
Jedoch kommt die Verwendung solcher Ansätze auch mit höherem organisatorischen Aufwand, denn es müssen Geschäftsbeziehungen zu mehreren Anbietern gepflegt werden
und die Verknüpfung der verschiedenen Produkte kann zusätzliche Kosten und Aufwand mit sich bringen.
Außerdem kann der Verbindungskanal zwischen Produkten von verschieden Anbietern ein Flaschenhals für den Informationsaustausch und eine Sicherheits-Schwachstelle sein.

\subsection{Überblick über Vor- und Nachteile}
\paragraph{Betrachtung in dieser Arbeit}
Für die letzliche Heranführung an die Herausforderung, Anbieterbindung, werden zunächst die verbreitetsten Vor- und Nachteile des Cloud-Computings dargesellt und kurz inhaltlich ausgeführt. Die Gesichtspunkte beschränken sich nicht zwangläufig auf einzelne Service-Modelle oder Architekturen.

\paragraph{Versprechungen und Vorteile}
Zu den Vorzügen zählen unter anderem Kosteneinsparung, verbesserte Skalierbarkeit, Wiederherstellugsmöglicheiten, Datensicherheit und weitere Punkte,
die im Kapitel 1.4 namens \glq utopische Versprechungen des Cloud Computings\grq\ im Buch Cloud Governance aufgeführt werden. \parencite{Mezzio2023}

Mit der aufgezeigten Popularität bei Unternehmen (vergleiche Abbildung \ref{fig:A_Nutz_CC}),lässt sich argumentieren, dass diese Versprechungen für Entscheidungstragende handfest sind.

\subparagraph{Kosteneinsparung}
Bei IT-Dienstleistern im GKV-Markt, die ihre Anwendungen selbst verwalten, fallen erfahrungsgemäß für Folgendes Kosten an:
\begin{itemize}
\item[-] Mitarbeitende für die Entwicklung, Betrieb und Wartung von Anwendungen
\item[-] Lizenzen für Kaufsoftware (auch für IDEs und Code-Verwaltungsplattformen)
\item[-] Zertifizierung (bei selbst hergestellter Software)
\item[-] externe zusätzliche Mitarbeiter bei großen Projekten
\item[-] externe Entwicklungs-Projekte
\item[-] Schulungen von Mitarbeitenden für neue Technologien und Anwendungen
\item[-] zusätzliche Mitarbeitende für (Personal-)Verwaltung, Buchhaltung und Einkauf
\item[-] zusätzliche Büroflächen
\end{itemize}
Soll darüberhinaus, auch die Infrastruktur selbst verwaltet werden, fallen zusätzlich noch Kosten für Folgendes an. 
\begin{itemize}
\item[-] Grundstücke, Gebäude und Räumlichkeiten
\item[-] Strom-, Wasser- und Wärmeversorgung
\item[-] Mitarbeitende für Installation, Betrieb und Wartung von Hardware
\item[-] Anschaffung, Ersatzteile und Entsorgung von IT-Komponenten
\item[-] Sonstiges wie beispielsweise Möbel
\item[-] Sicherheit wie Überwachungskameras, Sicherheitskräfte und Zertifizierungen
\end{itemize}
Besonders zu unterstreichen sind die initialen Investitionen, die in dieser Herangehensweise notwendig sind.

Durch diese eigene Darstellung der Kostensituation wird gezeigt, welche Kosten durch den Verzicht auf Cloud-Computing und die Zusammenarbeit mit SaaS-Anbietern entstehen können.

Des Weiteren wird ein Kontext für die Kosten von Cloud-Computing geschaffen.
Welche wiederrum bei der Integration von Cloud-Computing im Unternehmen teilweise oder vollständig anfallen.
Zu welchem Grad Kosten eigener Infrastruktur und Software wegfallen, hängt vom Service-Modell und der gewählten Architektur ab.

Grundsätzlich werden für die erfolgreiche Integration von Cloud-Computing Mitarbeitende mit neuen Kompetenzen benötigt. Zu diesen Kompetenzen zählt die Kontrolle, Regulierung und Überwachung der Dienstleistungsvereinbarungen (kurz: SLA) mit den Cloud-Computing-Anbietern und der Cloud-Computing-Nutzung durch das Unternehmen. \parencite{Mezzio2023}

Auch zu beachten ist, dass Schulungen für die neuen Bedienoberflächen der SaaS-Anbieter notwendig werden. Hat der gewählte Anbieter größtenteils eigene Software im Angebot, so gibt es entsprechend wenige Optionen für Schulungsanbieter auf dem Markt, was zu hohen Schulungskosten führen kann.

In der Summe lässt sich annehmen, dass finanzielles Risiko und große anfängliche Investitionen für Unternehmen wegfallen, und somit das Cloud-Computing eine geringere finanzielle Belastung sein kann. Jedoch ist ein vollständiger Kostenvergleich nicht Gegenstand dieser Arbeit, und soll hier nur ergänzend die Arbeit einleiten.
 
\subparagraph{Skalierbarkeit}
Bei herkömmlichen Architekturen mit Hardware und Software in eigner Verwaltung durch das Unternehmen oder den IT-Dienstleister ist die Herausforderung bezüglich der Skalierung erfahrungsgemäß der fehlende Vorrat freier Rechenressourcen, sowohl bei kurzfristiger Skalierung durch Lastspitzen, als auch bei langfristigen Skalierungen  beim Wachstum des Unternehmens.

Zur Verbesserung der Anschaulichkeit werden Beispiele der gesetzlichen Krankenkasse aufgeführt, was im Abschnitt \ref{vorstellung-aok} konkret vorgestellt wird.

Bei kurzfristiger Skalierung über den üblichen Arbeitstag, aber auch über das Arbeitsjahr gibt es Zeiten und Zeiträume, in denen mehr Kapazitäten benötigt werden.
Beispielsweise gibt es in der AOK Bayern und AOK PLUS ein Mitarbeiterportal für die Zeitbuchung und das Stellen von verschiedenen Anträgen wie Urlaubsanträgen. 
Hierbei gibt es sicherlich Lastspitzen am Morgen und am Nachmittag. 
Ein weiteres Beispiel ist die Kommunikationssoftware, 
die bei unternehmensweiten Besprechungen zwei- oder viermal im Jahr deutlich stärker ausgelastet wird.
Gibt es freie Ressourcen, die bei alltäglicher Auslastung im Leerlauf sind, dann können diese bei Bedarf dazu geschalten werden. Das bedeutet für die Endnutzer kürzere Wartezeiten bei der Benutzung von Anwendungen.

Im Gegensatz hierzu gibt es auch langfristige Skalierungen, wenn ein Unternehmen wachsen möchte. 
Bei der Krankenkasse AOK Bayern oder AOK PLUS ist starkes Wachstum in einem kurzen Zeitraum eher untypisch, 
da sich die Versichertenzahl üblicherweise gleichmäßig und allmählich erhöht.
Dennoch können neue Bedürftnisse im Bezug auf Rechen- oder Speicherleistung entstehen, 
wenn sich gesetzliche Vorgaben ändern oder neue Technologien auf den Markt kommen.
Gibt es hier freie Ressourcen, können ohne Ausfall alte Komponenten ausgetauscht werden oder ohne Austausch direkt mehr Ressourcen verwendet werden.

\subparagraph{Wiederherstellungsmöglichkeiten}
Bei technischen Fehl\-funktionen, Schä\-den durch Feu\-er oder Wasser, oder bei Verschlüsselung von Daten ist es wichtig eine Momentaufnahme der eigenen Systeme zu haben, 
um die verlorenen Informationen wiederherzustellen. 
Je aktueller die Momentaufnahme dabei ist, 
desto weniger Daten gehen durch den Vorfall und die Wiederherstellung verloren.
Je mehr verloren geht, desto höher der finanzielle Schaden, der dadurch verursacht wird.

Die Abbildung \ref{SecMaß-Nutzung} zeigt ein Balkendiagramm über das Thema IT-Sicherheitsmaßnahmen aus dem Jahr 2018. Neben anderen Maßnahmen wurde das Erstellen von Backups für Daten abgefragt und jedes der 503 Industrieunternehmen gab an, das es diese Maßnahme im Einsatz hatte.
Die Abbildung zeigt außerdem weniger weit verbreitete Maßnahmen wie Intrusion Detection System (IDS) oder die Verwendung von Penetrationstests. \parencite{StatistaSecMassnahmen2018}

Dennoch lässt sich schlussfolgern, dass die Relevanz von Backup- beziehungsweise Wiederherstellungsmöglichkeiten vollständig erkannt wurde. Doch auch die richtige und häufige Durchführung ist relevant.
%Wie im Abschnitt \ref{produkt-übersicht-saas} beschrieben, zählt zu den gängigen Produkten vieler SaaS-Anbieter ein Produkt zur Verwaltung und automatischen oder manuellen Erstellung von Sicherungen.
Durch die Werbung für das Produkt, könnten sich mehr Kunden, die auf SaaS setzen, mit dem Thema auseinandersetzen.

\begin{figure}
\begin{center}
\includegraphics[width=0.75\linewidth]{figures/statistic_id417433_umfrage-zu-vorbeugenden-technischen-it-sicherheitsmassnahmen-in-unternehmen-2018}
\end{center}
\label{SecMaß-Nutzung}\caption{Vorbeugende technische IT-Sicherheitsmaßnahmen}
\end{figure}

\subparagraph{Datensicherheit}
Das Thema Wiederherstellungsmöglichkeiten liefert einen Einstieg in die Art und Weise wie Saas-Anbieter Kunden zum Oberthema Datensicherheit heranführen können.
Durch das Angebot von diversen Produkten werden die Themen auch für Entscheidungstragende greifbarer, der Einstieg wird leichter und der Bedarf nach Fachpersonal sinkt.

Letzeres ist ein Punkt der ein großes Gewicht hat, denn Fachpersonal ist oftmals eine Hürde bei der Digitalisierung allgemein und damit auch speziell bei Sicherheitsmaßnahmen \parencite{StatistaHuerden2025}

Wird die Thematik in ein oder mehrere leicht bedienbare Produkte verpackt, für deren Bedienung nur wenige Mitarbeitende nötig sind, so steigt die Sicherheit durch die Verwendung von SaaS-Produkten.

\subparagraph{Weitere Punkte}
Außerdem werden im ursprünglich zitierten Kaptiel noch \glqq Vorteile der Cloud\grqq\ wie \glq Internet of Things\grq -Funktionalitäten,
ver\-besserte Zu\-sammen\-arbeit, Umwelt\-freund\-lich\-keit,
Mitarbeiter-Engagement, Echtzeit-Software-Updates und Analyse-Möglich\-keiten angesprochen. \parencite{Mezzio2023}

Da diese Punkte jedoch in der Gesamtaufstellung auf den hinteren Plätzen der Einordnung der Quelle landen, werden diese in diesem gemeinsamen Abschnitt kurz aufgeführt.

Besonders hervorzuheben ist der letzte Platz der Auflistung,
der wie auch bei anderen Vorteilen bereits, 
durch Produkte bei vielen Anbietern ermöglicht wird. 
Die Analyse von Daten, 
Netwerkverkehr oder sonstigen Informationen wird durch eigene Produkte für die Kunden von SaaS-Anbietern stark vereinfacht und zugänglicher gemacht.

\paragraph{Herausforderungen und Nachteile}
Die Herausvorderungen und Nachteile der Cloud werden im späteren Kapitel 3.6 \glq Der organisatorische Einfluss von Cloud-Computing\grq\ des Buches \glq Cloud Governance\grq\ aufgelistet:
\begin{itemize}
\item[-] Sicherheit (gegen Cyber-Angriffe)
\item[-] Kosten(-regulierung)
\item[-] (Integration von) Alt-Anwendungen
\item[-] Ausfälle
\item[-] Anbieterbindung
\item[-] (Verlust von) technischem Fachwissen
\end{itemize}
Die Auf\-zählung wurde aus dem Eng\-lischen über\-setzt und es wurde Kon\-text er\-gänzt.
\parencite[Kap.~3.6, S.~26]{Mezzio2023}
Die konkreten Punkte stammen aus einem Blog-Artikel der IT-Sicherheits\-firma Conosco.
\parencite{Conosco2020}

\subparagraph{Sicherheit}
Laut einem auf der Seite von Conosco aufgeführten Studie gaben drei Viertel der befragten Unternehmen 2020 an, 
dass sie starke oder enorme Bedenken bezülich der Sicherheit von Public-Cloud-Computing haben.
Strenge Gesetzesauflagen zum Thema Datenschutz und Sicherheit (vergleiche Abschnitt \ref{legal}) steigern diese Verunsicherung,
wird in der Veröffentlichung der IT-Sicherheitsfirma weiter erklärt. \parencite[Abs.~1]{Conosco2020}

Zusätzliche Hintergründe liefert der Report aus dem Jahr 2025 von Statista nach dem 72\% angaben, dass \glqq Vertrauen in Sicherheit und Compliance des Cloud-Providers\grqq\ zentral bei der Auswahl des Anbieters sind. \parencite[S.~35]{StatistaReport2025}

Dabei hervorzuheben ist jedoch, dass es einen Unterschied zwischen dem Aufkommen von Sicherheitsbedenken in der Auswertung bei Conosco und dem Auswahlkriterium, das in dem Statista Report gefordert wird.

Es lässt sich argumentieren, dass es eine Gruppe von Unternehmen gibt, die tatsächliche Sicherheitsbedenken haben und eine andere Gruppe, die ein gewissen Level von Sicherheitszertifizierung fordern. Die beiden Gruppen sind nicht zwangsläufig deckungsgleich.

Die Anbieter versuchen Sicherheit auszustrahlen und werben mit dedizierten Sicherheitsmitarbeitenden und Zertifizierungen.
Deutlich wird dies beispielsweise am Werbebanner, 
das bei vielen Produktseiten eingebaut ist (vergleiche \parencite[Produkt:~Azure~Backup, Abs.~Eingebettete Sicherheit und Compliance]{AzureKat2026}). 
An welche Zielgruppe diese Werbung gerichtet ist, 
ist fragwürdig, 
da aus technisch-fachlicher Sicht konkrete kryptographische Algorithmen, Statistiken über Angriffsabwehr und konkrete Zertifizierungen aussagekräftiger als kontextfreie Zahlen wie die absolute Anzahl der Zertifizerungen. 
Eine tiefere Beschäftigung mit \glq Security Assurance \grq oder Sicherheit der Cloud-Computing-Dienstleister ist nicht Gegenstand dieser Arbeit. (vergleiche \parencite{SecAssurance2026})

\subparagraph{Kostenregulierung}
Neben der Sicherheit ist auch der gewisse Kontrollverlust über die Kosten eine Herausforderung beim Übergang zu Cloud-Computing-Dienstleistern. 
Die integrierte automatische Skalierung von Ressourcen am aktuellen Bedarf muss insofern einschränkt werden, 
dass bei zufällig oder absichtlich herbeigeführten Lastspitzen keine unerwartet hohen Kosten entstehen.
Es lässt sich anschaulich argumentieren, 
dass ein Onlineshop ein Interesse an der automatischen Skalierung der Ressourcen hat, 
wenn mehr Käufer als üblich die Seite besuchen (beispielsweise im der Weihnachtszeit) \parencite[S.~17]{Hennrich2023}.
Die Konkretisierung der technische Umsetzung der Skalierung und die konkret notwendigen Ressourcen sind hierfür nicht relevant. 
Stattdessen ist hervorzuheben, 
dass die Ressourcen mit einem gewissen Preis pro Einheit versehen sind (vergleiche Abschnitt \ref{abrechnungsmodelle}).
Im Gegensatz hierzu ist es denkbar, dass es illegitime Situationen wie DDoS-Angriffe gibt, in denen die Ressourcen nicht mit den absichtlichen und bösartigen Lastspitzen provoziert werden, um wirtschaftlichen Schaden durch Aussetzen der Verfügbarkeit für legitime Zugriffe oder unkontrollierte Skalierung der Ressorucen herbeizuführen.

\subparagraph{Weitere Punkte}
Neben den größten Herausforderungen kann die Adaption von Cloud-Computing zu Kompitabilitätsproblemen zu alten Anwendungen führen. 
Beispielsweise ist es möglich, 
dass die Dienstleistungen und die Software der Cloud-Computing-Anbieter keine geeignete Plattform für alte Anwendungen haben. 
In diesem Fall kann durch zusätzliche Anpassungen an der alten Anwendung, 
die Ablöse durch einen Dienst des CC-Anbieters oder die Neuentwicklung der alten Anwendung in kompatibler Art und Weise die Probleme lösen.
Inkompatible bestehende Anwendungen sorgen jedenfalls jedoch durch notwendige Anpassungen für zusätzliche Aufwände bei der Migration.
Der meist starre Leistungskatalog der Anbieter, der die ununterbrochene Weiternutzung von alten Anwendungen durch individuelle Untersützung durch den Anbieter nicht vorsieht, wird auch in der Literatur angesprochen. (vergleiche \parencite[S.~40f]{Hennrich2023})

Problematisch ist im Kontext auch der Verlust von Fachwissen, wobei anzumerken ist, dass die Unternehmen dadurch auch Personalkosten einsparen können.
Im Zusammenhang mit dem Vorteil \glq Kosteneinsparung\grq\ lässt sich argumentieren, 
dass der Wegfall von hochqualifiziertem Fachpersonal und dem dazugehörigen Wissen ein Teil der versprochenen Kosteneinsparungen sind. 
Inwiefern diese Einsparungen sich langfristig rechnen, wenn kompliziertere Anpassungen oder Migrationen notwendig werden, soll in dieser Arbeit nicht diskutiert werden.
Neben Verlust von Fachwissen kann auch die Abwesenheit von Fachwissen im Bezug auf Cloud-Computing problematisch werden. Diese Sichtweise wird von Conosco als weitere Herausforderung herausgearbeitet. Qualifizierte Fachkräfte für die Auswahl des richtigen Anbieters zu finden, ist laut der Veröffentlichung der Firma schwierig. \parencite[Abs.~6 Technical Knowledge]{Conosco2020}


Ausfälle der Dienste von CC-Anbietern sind gleichermaßen möglich wie der Ausfall von \glq On-Premise\grq -Infrastruktur. Ausfälle können sowohl unabsichtlich als auch absichtlich entstehen.
Zu beachten ist jedoch der Kontrollverlust auf Unternehmensseite und das höhere Schadensrisiko (vergleiche \parencite[Abs.~4 Downtime]{Conosco2020})
Bei Ausfällen von Cloud-Computing-Rechenzentren sind, im Gegensatz zu unternehmenseigenen Rechenzentren, viele Unternehmen betroffen.
Der Schaden durch einen einzelnen Vorfall steigt.
Dieses größere bekanntere Ziel in Kombination mit dem höheren Schaden, kann zu erhöhtem Interesse bei Angreifern führen (vergleiche absichtliche Ausfälle und Herausforderung: Sicherheit).

\paragraph{Einordnung der Quellen für Vor- und Nachteile}
Bei der Recherche vom richtigen Umgang mit der Migration zu Cloud-Computing-Anbietern ist es elementar zu beachten, 
dass viele Quellen Herausforderungen und Chancen auf Basis von Erfahrungen und vor dem Hintergrund veröffentlichen, 
dass diese auch kostenpflichtige Schulungs- und Beratungsdienste für die Migration anbieten.
Die Aussagen verschiedener Quellen sind im Wesentlichen übereinstimmend, wodurch sich die notwendige Seriösität für diese Arbeit nachweisen lässt.

\subsection{Detailbetrachtung der Anbieterbindung}
\label{anbieterbindung}
\paragraph{Definitionen von Anbieterbindung}
Einführend in die Betrachtung der Herausforderung werden Erklärungen und Definitionen des Phänomens aus der Literatur betrachtet. 
\begin{quote}
Zu den meistgenannten Risiken zählen Fragen der Abhängigkeit von einem Anbieter, die im Worst Case zu einem klassischen Vendor-Lock-in führen können.
Auch besondere IT-Konfigurationen einer Cloud können zu einem faktischen Lock-in führen, wenn diese Konfiguration von keinem anderen Anbieter oder nur mit erheblichem wirtschaftlichem Aufwand bereitgestellt werden kann.
Eine Abhängigkeit kann auch im Bezug auf die Verfügbarkeit der IT-Ressourcen bestehen, insbesondere bei einem Ausfall und nicht vorhander Redundanzkonzepte. 
Die Qualität der Datenanbindung und die Bandbreite der Leitungswege (Netzanbindung bzw.\ Connectivity) kann dies ebenfalls betreffen. Gleiches gilt für allgemeine Fragen der Interoperabilität, insbesondere beim Einsatz von Cloud-Infrastrukturen mehrer Anbieter im Rahmen von Multi-Cloud-Szenarien.
\parencite[Kap.~2.4,S.~33f]{Hennrich2023}
\end{quote}

Ergänzend dazu steht die Erklärung \glqq [Anbieterbindung] tritt auf, wenn eine Organisation, die ein Cloud-Computing-Produkt oder -Dienstleistung nutzt, nicht einfach zu einem konkurrierenden Anbieter wechseln kann.\grqq\ \parencite[S.~82]{Mezzio2023}

Der IT-Sicherheitsexperte Conosco vergleicht Cloud-Computing-Anbieterbindung im gewerblichen Kontext für Unternehmen mit der Bindung, 
die Endkonsumenten im privaten Kontext bei Technologie-Unternehmen wie Apple, 
die \glqq Anwendungen, Alleinstellungsmerkmale (Features) und Geräte, 
die nur mit anderen Apple Produkten kompatibel sind, 
auf den Markt bringen\grqq .
Gleichermaßen \glqq fällt es schwer Anwendungen von einem Cloud-Computing-Anbieter zu übertragen, wenn das notwendig wird.\grqq \parencite[Frei aus dem Englischen übersetzt, Abs.~5]{Conosco2020}.

\paragraph{Folgerungen aus den Definitionen}
\label{komponenten-anbieterbindung}
Vendor-Lock-In (Anbieterbindung) hat drei wesentliche Komponenten.
\begin{itemize}
\item[-] Technische Ursachen
\item[-] Organisatorische Verstärkung
\item[-] Finanzielle Verstärkung
\end{itemize}
Für den Rahmen dieser Arbeit wird argumentiert, dass die Anbieterbindung ihre Wurzeln im Fehlen von technischer Interoperabilität hat. 
Die fehlende Kompatibilität wird durch organisatorische Probleme wie beispielsweise SLAs, 
die nicht mit passender Exit-Strategie verhandelt wurden, verstärkt. Gleichermaßen verstärkt die Abwesenheit von finanziellem Spielraum für Migrationen, 
wenn die notwendig werden, den Vendor-Lock-In.

Es wird argumentiert, dass die Lösung oder das Umgehen von Anbieterbindungen bei den technischen Ursachen beginnt und eine die Minimierung technischer Abhängigkeiten das Fundament für eine minimale Anbieterbindung ist.

\paragraph{Technische Ursache für Anbieterbindung}
Die Abbildung \ref{fig:technUr_Anbieterbindung} skizziert, wie das Paradigma Cloud-Computing und im Detail das SaaS-Service-Modell Anbieterbindung hervorbringt.
Es werden Abhängigkeiten in der CC-Umgebung schematisch dargestellt,um zu visualisieren, welche Verbindungen bei einem Anbieterwechsel neu geknüpft werden müssten.

In der Abbildung befinden sich die vier Kernelemente Kunde (1), Anbieter (2), Rechenressourcen (3) und Daten (ohne Nummerierung). Im Zentrum steht dabei der Anbieter, vertreten durch seine Programme.

Die eigene Darstellung soll zudem verdeutlichen, an welchen Stellen Kompatibilität zwischen Anbietern aus Sicht des Kunden wichtig wird. Die entsprechenden Punkte sind rot hinterlegt.
Es wird argumentiert, dass generell die Schnittstellen zwischen dem Anbieter und dem Personal (GUI) oder den Programmen (API) des Unternehmens kritische Punkte sind. 
An diesen Stellen ist eine Kompatibilität zum neuen Anbieter notwendig, 
damit ein Wechsel möglichst geringe Aufwände mit sich bringt. 
Zudem Relevant sind die vom Anbieter produzierten Daten,  beispielsweise von unternehmensbezogenen Auswertungen, 
die nach einem Wechsel mit den Anwendungen des neuen Anbieter funktionieren müssen.

\begin{figure}
\begin{center}
\includegraphics[width=0.75\linewidth]{figures/technUr_Anbieterbindung}
\end{center}
\label{fig:technUr_Anbieterbindung}\caption{Eigene Skizze der Beziehung zwischen Cloud-Computing-Kunde, CC-Anbieter und den Rechenressourcen}
\end{figure}

\section{Motiviation}

Die Motivation unterstreicht im Kontext der Arbeit die Notwendigkeit zur Erfüllung der Ziele und umreißt kurz die Gründe für die Themenwahl.

\subsection{Besonderheiten des Themas}

\paragraph{Unterschiede zu anderen Herausforderungen der Cloud}
Im Vergleich zu Kostenanalyse oder Gegenüberstellung der Besonderheiten von unterschiedlichen Cloud-Computing-Anbietern, 
sind die technischen Ursachen der Anbieterbindung weniger greifbar und schlechter quantifizierbar.
Damit das Thema für Kunden von CC-Dienstleistern greifbar ist, 
müsste bei jedem Produkt die Kompatibiltät zu alternativen Produkten auf dem Markt aufgelistet werden,
wie beispielsweise auch die Preislisten in branchenüblichen Mengenangaben aufgeführt werden. So wären direkte unkomplizierte Vergleiche möglich, die Auflistung existiert allerdings nicht (vergleiche Abschnitt \ref{produkt-übersicht-saas}).
Aufgrunddessen ist die Entwicklung einer Analysemethode besonders interessant.

\subsection{Bedarf einer Analyse von Anbieterbindung}

\paragraph{Strategische Bedeutung für Unternehmenen}
Die Beschäftigung mit Anbietern ist spannend, denn sie hat strategische und politische Komponenten.
Die Wahl eines Cloud-Anbieters für ein Unternehmen ist elementar und Anbieterbeziehungen durchlaufen einen Lebenszyklus (vergleiche Software-Lebenszyklus).
Obwohl es Diskrepanzen zwischen der Praxis und der Theorie gibt, so sollte schon bei der Schließung einer neuen Geschäftsbeziehung deren Ende und Wechsel-Strategie festgelegt sein.
Hierfür ist die Durchleuchtung eines Anbieters hinsichtlich Vendor-Lock-In schon im Voraus wichtig.
Wie schon festgestellt, wird die Anbieterbindung beim Beenden einer Geschäftsbeziehung relevant.
Dafür zentral ist, wann das Ende der Geschäftsbeziehung in einem Unternehmen erreicht ist.
Bei alleinstehenden Anwendungen beispielsweise wird die Lebenszeit überlicherweise auf eine gewisse Jahreszahl begrenzt. 
Allerdings können wie auch bei den klassischen  Anwendungen bei einem Cloud-Anbieter Bedingungen eintreffen, die einen früheren Wechsel verlangen.

Diese Bedingungen können finanzieller Art sein. So könnte etwa der aktuelle Anbieter in Anbetracht seiner Leistungen nicht mehr wirtschaftlich sein.

Nicht nur finanzielle Aspekte können zu einem Wechselwunsch beim Kunden führen.

Durch Anpassungen am Leistungskatalog und die vertragliche Möglichkeit manche Leistungen nicht mehr anzubieten, kann es dazu kommen, dass notwenige Bausteine nicht mehr vom Cloud-Computing-Anbieter unterstützt werden.
Solche Anpassungen sind aufgrund der festen Vertragsregeln zwar nie plötzlich, meistens aber ein Argument für einen Wechsel.

Außerdem kann es dazu kommen, 
dass Kunden von mehreren Anbietern ihre benötigten Leistungen auf einen einzigen konsolidieren wollen oder im Gegenbeispiel ihre Anforderungen auf mehrere Anbieter verteilen wollen,
um die unterschiedlichen Alleinstellungsmerkmale mehrerer Anbieter gleichzeitig zu nutzen.

Zuletzt kann es auch durch äußere Faktoren wie gesetzliche Vorgaben,
denen das Produkt des aktuellen Anbieters nicht mehr folgt, dazu kommen,
dass ein Wechsel unbedingt notwendig wird.
Auch geopolitische Änderungen wie Zölle oder Gesetze zählen zu den Gründen für das frühzeitige Ende der Geschäftsbeziehung.

%\paragraph{Strategische Bedeutung auf geopolitscher Ebene}
%TODO optional

\section{Abgrenzungen}
In der Abgrenzung werden im Kontext dieser Arbeit gezielt und explizit Inhalte ausgeschlossen, um die Zielsetzung möglichst geradlinig und anwendungsbezogen zu erfüllen.
\subsection{Distanzierung von ökonomischen Ansätzen}
Im Gegensatz dazu sind vertragliche oder ökonomische Kriterien Gegenstand dieser Arbeit.
Zur Verdeutlichung wird also beispielsweise nicht untersucht, ob die These, dass das Nutzen eines teurerer Cloud-Computing-Anbieter seltener zum Vendor-Lockin führt, zutrifft.

%\subsection{Fokus auf öffentliche Cloud}
%TODO
\subsection{Eingrenzung der zu analysierenden Anbieter}
\label{lead-cc-A}
\paragraph{Herangehensweise}
Über verschiedene Kennzahlen wie dem Marktanteil lassen sich die führenden Cloud-Computing-Anbieter ermitteln. Aus der Auswertung von Canalys im Rahmen des Statista Berichtes über Cloud-Computing ergibt sich, dass rund 65\% des Marktes im Besitz von nur drei Anbietern sind.
\begin{quote}
Microsoft Azure erzielte im Cloud-Markt für Infrastruktur-Services im 2. Quartal 2025 einen Umsatzanteil von rund 22 Prozent. 
Marktführer ist Amazon Web Services (AWS) mit einem Marktanteil von über 30 Prozent. 
Der weltweite Umsatz im 2. Quartal 2025 betrug laut Quelle 95,3 Milliarden US-Dollar. 
Im Gesamtjahr 2024 belief sich der weltweite Umsatz mit Cloud Computing auf rund 596 Milliarden US-Dollar – Tendenz steigend. 
Von der wachsenden Nachfrage profitieren insbesondere die führenden Anbieter Microsoft, Amazon und Google.
\parencite[S.~7]{StatistaReport2025}
\end{quote}
Die übrigen Anteile des Marktes beinhalten ebenfalls große internationale Unternehmen.
Darunter zählen IBM und Alibaba mit jeweils 4\% der Anteile des PaaS-Marktes oder Oracle mit 5\% des PaaS-Marktes im Jahr 2023. \parencite[S.~22]{StatistaReport2025}

Diese Arbeit fokussiert sich schwerpunktmäßig auf die größten Anbieter, wobei auch kleinere Anbieter erwähnt werden. Dabei wird auf Basis der Analyseergebnisse diskutiert, welche Bedeutung die Position auf dem Markt für die Anbieterbindung hat.

\paragraph{Kernelemente der Produktkataloge von Anbietern}
Die erarbeiteten Kernelemente basieren auf den Produktkatalogen der Anbieter Microsoft, Google und Alibaba. (vergleiche \parencite{AzureKat2026,GoogleKat2026,AlibabaKat2026})

\subparagraph{versteckte Duplikate}
Die Anbieter nutzen eine Vielzahl von Kategorien und Produkten. Teils wiederholen sich Produkte in verschiedenen Kategorien.
Dies gilt beispielsweise für das Produkt \glq Azure DevOps\grq\ , welches in der Kategorie \glq Entwicklertools\grq\ und \glq DevOps\grq\ vorhanden ist oder dem Produkt \glq Azure Data Lake Storage\grq\ , 
welches sowohl in der Kategorie \glq Speicher\grq\ ,als auch in der Kategorie \glq Analysen\grq\ zu finden ist. \parencite{AzureKat2026}
Für die generelle nutzerfreundliche Navigation und auch im Kontext dieser Arbeit ist dieser Umstand ungünstig.

\subparagraph{unklare Beziehungen}
Unter den Produkten der Anbieter befinden sich im Katalog einerseits ganze Plattformen und andererseits Produktbausteine.
Exemplarisch dafür steht die Kategorie \glq Sicherheit\grq\ , die bei allen Anbietern eine große Menge von Sicherheits-Produkt beinhaltet. Aus fachlicher Sicht, lässt sich jedoch annehmen, dass eine Plattform zur Verteidigung der Systeme gegen Bedrohungen mit entsprechenden Werkzeugen zur Verwaltung von diveren Sicherheitsmaßnahmen (wie (Web-Application-) Firewalls, IDS und Proxies) ausgestattet ist.
Gegen diese Annahme werden solche Produkte jedoch seperat aufgeführt.\parencite{GoogleKat2026}

\paragraph{aufwendige Navigation}
Die Produktkataloge lassen sich wegen dem Umfang, den Duplikaten und der generellen Listenstruktur aufwendig navigieren. 
Beim gezielten Suchen stört der komplexe Aufbau der Internetseiten mit unübersichtlichen Bannern, Auswahllisten.
Illustriert wird dies durch die Produktseiten von Google Cloud, die zwei linksbündige und zwei rechtsbündige Kopfzeilen und eine Seitenleiste besitzen. 
Der eigentliche Inhalt, die Produktinformationen sind in verschiedene Boxen und Seitenabschnitte verstreut.
\parencite[Produkt:~Security-Command-Center]{GoogleKat2026}
Redundante Informationen wie übergreifende und produktunabhängige Daten blähen die Unterseiten auf. 
Dies zeigt sich besonders bei dem Aufbau der Produktseiten von Microsoft Azure. 
Diese sind unter Anderem gefüllt mit wenigen generischen Fotografien, die sich über den gesamten Katalog wiederholen.
\parencite[Produkt:~API-Management]{AmazonKat2026}

\paragraph{Relevanz für die Untersuchung}
Der kurze Überblick über die Produktkatalog ist ein Exkurs, der Herausforderungen bei der Umsetzung der Arbeit einleiten soll.
Außerdem ist es Gegenstand der Arbeit auf Basis von Produktkatalogen Rückschlüsse zu ziehen. 
Daher ist die Beschäftigung mit der Informationswiedergabe im Rahmen dieser Arbeit interessant.