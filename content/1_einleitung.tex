\chapter{Einleitung}\label{ch:einleitung}

\section{Heranführung}

\subsection{Theoretische Hintergründe der Cloud}

\paragraph{Definitionen der Cloud Computing}
Es ist zusammen mit der künstlichen Intelligenz einer der wohl schillerndsten und meistverwendetsten Begriffe in der \Gls{IT}-Branche wie der Fachbuchautor Thorsten Hennrich in seinem Fachbuch zu Cloud Computing nach der Datengrundschutzverordnung einleitend formulierte. \cite{Hennrich2023}

Pragmatischer ist die ISO Norm, nach der Cloud Computing ein Paradigma,um einen netzwerkbasierten Zugang auf ein skalierbares und elastisches Reservoir gemeinsam nutzbarer physischer oder virtueller Ressourcen nach dem Selbstbedienungsprinzip und bedarfsgerechter Administration zu ermöglichen. \cite{ISO22123}

Ergänzend dazu steht der Artikel vom US-amerikanischen Technologieunternehmen Microsoft mit dem Titel \glqq Was ist die Cloud?\grqq. 
Der Webartikel klärt aus Sicht des Cloud-Anbieters auf und definiert die Cloud als \glqq […] Online-Speicherplatz, in dem Personen und Unternehmen ihre Dateien und Anwendungen speichern, die von überall mit einer Internetverbindung zugänglich sind.\grqq
Des Weiteren bietet die Cloud laut dem Anbieter \glqq Dienste wie Rechenleistung, Datenbanken, Netzwerke und Softwareanwendungen.\grqq
\cite{CloudDefMicro2025}

\paragraph{Folgerungen aus den Definitionen}
Aus diesen Perspektiven zur Technologie lassen sich wichtige Grundlagen wie den Zusammenhang zwischen Internet und Cloud und die inherente Flexibilität als Fundament der Technologie ableiten.

IT-Kompontenen wie der Speicher für Daten, die zu jeder Organisation und jedem Unternehmen gehören und traditionell am gleichen Standort beziehungsweise \glq \Gls{on-Premise}\grq\ aufzufinden waren, können ausgelagert werden. 
Je nach Umsetzungsart sind die Komponenten der Cloud trotzdem auf dem gleichen Grundstück, im gleichen Land oder an einem völlig anderen Ort auf der Erde. 
Genutzt werden die Komponten jedoch immer per Fernzugriff über das Internet.

Diese Ausgestaltung alleine ermöglicht eine höhere Flexibilität, da Mitarbeitende des Unternehmens aus der Ferne auf die Komponenten zugreifen können. 
Darüberhinaus ist ein wichtiger Bestandteil des Konzeptes die Skalierung der Ressourcen. 
Die Skalierung beinhaltet, dass zusätzliche Komponenten auf Wunsch eines Unternehmens zugeschaltet werden können.

\paragraph{Begriffsunterscheidung Cloud und Cloud-Computing}
Spannend ist, 
dass Cloud und Cloud-Computing oft synonym verwendet werden. So auch im Informationsmaterial von Microsoft,
wo das Konzept von Cloud-Computing beschrieben wird, 
aber kurz als \glq Cloud \grq abgekürzt wird.
Zur Verwirrung trägt erschwerend bei, 
dass Microsoft einen zweiten Artikel mit dem Titel \glqq Was ist Cloud-Computing?\grqq veröffentlicht hat, 
der eine sehr ähnliche Definition beinhaltet, aber die Cloud selbst als Bezeichnung für das Internet beschreibt. \cite{CloudDefMicro20252}
Diese Arbeit beschränkt sich auf Cloud-Computing. 

\paragraph{Verbreitung der Technologie}
Die Definitionen geben einen Einblick in das vermeintliches Potential der Technologie.

Statistische Befragungen der letzten Jahrzehnte von Führungskräften deutscher und internationaler Unternehmen, machen deutlich, dass das Thema die Aufmerksamkeit der Führungsebene längst erreicht hat und Cloud-Computing in die Unternehmen eingezogen ist.

Dies geht aus Abbildung \ref{A_Nutz_CC} hervor.
Die Abbildung zeigt ein Balkendiagramm,
das die Nutzung von Cloud-Computing über die Jahre 2011 bis 2024 beschreibt.
Hierbei zeigt die y-Achse die Prozentzahl der Befragten in Prozent und die x-Achse die Jahreszahlen.
Ein Datenpunkt pro Jahr zeigt den Anteil der Unternehmen, die bereits Cloud Computing nutzen und ein weiterer Datenpunkt zeigt den Anteil der Unternehmen, die den Einsatz noch planen.
Laut der Visualisierung stieg im Zeitraum der Untersuchung,
welcher sich auf 13 Jahre beläuft, die Nutzung von Cloud-Compting von 28% auf 98%. 
Der Anteil der Befragten, die angaben,
dass sie die Technologie nur planen, sank von 22% auf 0%. 
Für die Untersuchung wurden Geschäftsführer und IT-Führungskräfte aus 503 Unternehmen befragt.
Im Kontext dieser Arbeit gibt die Befragung einen ersten Einblick in die Nutzungstrends der Technologie an.
Der hohe Nutzungsanteil zum Ende der Datenreihe,
lässt darauf schließen, dass sich die Entscheidungstragende heute nicht mehr fragen, ob sie diese Technologie einsätzen,
sondern welche Anbieter wie genutzt werden.

\begin{figure}
\begin{center}
\includegraphics[width=0.75\linewidth]{figures/statistic_id177484_umfrage-zur-nutzung-von-cloud-computing-in-deutschen-unternehmen-bis-2024}
\end{center}
\label{A_Nutz_CC}\caption{Nutzung von Cloud Computing}
\end{figure}

\paragraph{Funktionsweise und gängige Architekturen}

\paragraph{Versprechungen und Vorteile der Cloud}
Zu den Vorzügen zählen daher Kosteneinsparung, verbesserte Skalierbarkeit, Wiederherstellugsmöglicheiten, Datensicherheit und weitere Punkte,
die im Kapitel 1.4 namens "utopische Versprechungen des Cloud Computings" im Buch Cloud Governance aufgeführt werden. \cite{Mezzio2023}

Mit der Popularität (vergleiche Abbildung \ref{A_Nutz_CC}) wird klar, dass diese Versprechungen für Entscheidungstragende handfest.

\subparagraph{Kosteneinsparung}
Bei Unternehmen, die ihre Anwendungen selbst verwalten, fallen erfahrungsgemäß für Folgendes Kosten an:
\begin{itemize}
\item[-] Mitarbeitende für die Entwicklung, Betrieb und Wartung von Anwendungen
\item[-] Lizenzen für Kaufsoftware (auch für IDEs und Code-Verwaltungsplattformen)
\item[-] externe zusätzliche Mitarbeiter bei großen Projekten
\item[-] externe Entwicklungs-Projekte
\item[-] Schulungen von Mitarbeitenden für neue Technologien und Anwendungen
\item[-] zusätzliche Mitarbeitende für (Personal-)Verwaltung, Buchhaltung und Einkauf
\item[-] zusätzliche Büroflächen
\end{itemize}
Soll darüberhinaus, auch die Infrastruktur selbst verwaltet werden, fallen zusätzlich noch Kosten für Folgendes an. 
\begin{itemize}
\item[-] Grundstücke, Gebäude und Räumlichkeiten
\item[-] Strom-, Wasser- und Wärmeversorgung
\item[-] Mitarbeitende für Installation, Betrieb und Wartung von Hardware
\item[-] Anschaffung, Ersatzteile und Entsorgung von IT-Komponenten
\item[-] Sonstiges wie beispielsweise Möbel
\item[-] teilweise: Sicherheit wie Überwachungskameras und Sicherheitskräfte
\end{itemize}
Durch diese unvollständigen und schematischen Kostenaufstellung soll gezeigt werden, welche Kosten durch den Verzicht auf Cloud-Computing und die Zusammenarbeit mit SaaS-Anbietern entstehen können. 


\subparagraph{Skalierbarkeit}

\subparagraph{Wiederherstellungsmöglichkeiten}

\subparagraph{Datensicherheit}

\paragraph{Kapitalisierung der Cloud}
Aus dem Konzept der Cloud wird durch die Anbieter ein Produkt oder ein Katalogs mit Produkten. 
Der Bedarf nach den Leistungs\-versprechungen der Cloud ist enorm.
Nicht nur das Trainieren, sondern auch das Nutzen von KI-Modellen auf der eigenen Hardware ist rechenaufwändig. Auch andere Anwendungen wie das Managment von Kunden und Unternehmensressourcen wird immer komplexer.

Es gibt viele verschiedene Konfigurationen der Cloud.

Wie auch bei eigenen Rechenzentren aus einer Vielzahl von Architekturen und Marken gewählt werden kann,
so gibt es bei der Wahl der Cloud Liefermodelle und Produktbausteine, 
die nach den Bedürftnissen des Kunden eingekauft werden können.
Grundlegend kann gewählt werden wie viele Schichten des ursprünglichen Rechenzentrums in die Cloud gehoben werden sollen. Die entsprechenden Stufen hierzu sind ebenfalls in der ISO-Norm beschrieben und lauten (sortiert nach aufsteigender Kompetenzen des Anbieters):
\begin{itemize}
\item[-] Infrastructure as a Service
\item[-] Platform as a Service
\item[-] Software as a Service
\end{itemize} \cite{ISO22123}

\paragraph{Schematische Gegenüberstellung von Kosten und Funktionsweise}

\paragraph{Herrausforderungen und Nachteile der Cloud}
Die Herausvorderungen und Nachteile der Cloud werden im späteren Kapitel 3.6 "Der organisatorische Einfluss von Cloud-Computing" des Buches "Cloud Governance" aufgelistet:
\begin{itemize}
\item[-] Sicherheit (gegen Cyber-Angriffe)
\item[-] Kosten(-regulierung)
\item[-] (Integration von) Alt-Anwendungen
\item[-] Ausfälle
\item[-] Anbieterbindung
\item[-] (Verlust von) technischem Fachwissen
\end{itemize}
Die Aufzählung wurde aus dem Eng\-lischen übersetzt und es wurde Kon\-text ergänzt.
\cite{Mezzio2023}
Die konkreten Punkte stammen aus einem Blog-Artikel der IT-Sicherheitsfirma Conosco.
\cite{Conosco2020}
Der Fokus dieser Arbeit ist die Anbieterbindung.
Eine tatsächliche Anbieterbindung ist nur auf der Ebene Software as a Service möglich.

\subsection{Grundlagen der Anbieterbindung}

\paragraph{Definitionen von Anbieterbindung}
Vendor-Lock-In (dt. Anbieterbindung) ist ein Umstand in der Beziehung zu einem Anbieter aus Kundensicht. 
Dieser Umstand wird beim Beenden der Beziehung problematisch.
Denn möchte ein Unternehmen den Anbieter wechseln,
so ist eine Anbietermigration nötig.

Grundsätzlich ist die Migration von einem Anbieter zu einem Konkurrenten immer mit Aufwand verbunden,
wenn beispielsweise die Daten einer CRM-Anwendung des einen Anbieters zum Anderen gesendet werden müssen.
Problematisch wird es dann,
wenn die Daten in unterschiedlichen Formaten abgespeichert sind. 
Hilfreich sind dann Werkzeuge zur Migration,
welche bei branchenüblichen CRM-Anwendungen leicht erhältlich sind. 
Waren beim ursprünglichen Anbieter jedoch Anbieter-eigene Lösungen im Einsatz,
steigert sich der Migrationsaufwand über das zu erwartende Pensum hinaus.

Ein besonders schwerwiegender Vendor-Lock-In liegt vor, wenn Bausteine des Produkts gar nicht vom neuen Anbieter abgebildet werden können.

\paragraph{Grundsätzlich anfällige Cloud-Produkte}
In niedrigeren Liefermodellen wie Platform as a Service,
wo die Anwendungen und Daten in der Hand des Kunden liegen, ist Anbieterbindung generell kein Problem.
Dadurch,
dass die Anwendungen bei allen Anbietern gleichermaßen durch den Kunden gewählt und betrieben werden,
können diese Anwendungen zum neuen Anbieter einfach migriert werden.

Dies ist ebenfalls aufwendig, vor allem wenn die Umgebung beim neuen Anbieter anders ist. Jedoch ist eine Migration pauschal immer möglich.

Die Kunden habe dazu das technische Fachwissen im Hause,
um die Anwendungen entsprechen zu modifzieren oder umzukonfigurieren,
damit diese in der neuen Umgebung funktionieren.

\paragraph{Einordnung der Kritikalität des Problems}


\paragraph{Erkennung von betroffenen Cloud-Produkten}

\paragraph{Werkzeuge zur Messung von Anbieterbindung}
Zudem lassen sich Vorteile wie der Kostenpunkt leicht quantifizieren und damit vergleichen.
Auch die Skalierbarkeit lässt sich durch die Zeit messen, die es benötigt zusätzliche Hardware einzubinden, wenn beispielsweise hoher Verkehr es fordert.
Ferner lassen sich auch andere Eigenschaften der verschieden Cloud-Produkte wie die Anzahl der Backups oder die Anzahl von Datenlecks gegenüberstellen.


\subsection{Überblick über führende Cloud-Computing-Anbieter}
\label{lead-cc-A}
\subsection{Überblick über Cloud-Produkte}
\label{produkt-übersicht-saas}

\paragraph{Kernelemente der Produktkataloge von Anbietern}

\paragraph{Navigieren von Produktkatalogen}

\section{Motiviation}

\subsection{Bedarf eines Prozesses zur Einordnung von Anbieterbindung}

\paragraph{Strategische Bedeutung für Unternehmenen}
Die Beschäftigung mit Anbietern ist spannend, denn sie hat strategische und politische Komponenten.
Die Wahl eines Cloud-Anbieters für ein Unternehmen ist elementar und Anbieterbeziehungen durchlaufen einen Lebenszyklus (vergleiche Software-Lebenszyklus).
Obwohl es Diskrepanzen zwischen der Praxis unter der Theorie gibt, so sollte schon bei der Schließung einer neuen Geschäftsbeziehung deren Ende und Wechsel-Strategie festgelegt sein.
Hierfür ist die Durchleuchtung eines Anbieters hinsichtlich Vendor-Lock-In schon im Voraus wichtig.
Wie schon festgestellt, wird die Anbieterbindung beim Beenden einer Geschäftsbeziehung relevant.
Dafür zentral ist, wann das Ende der Geschäftsbeziehung in einem Unternehmen erreicht ist.
Bei alleinstehenden Anwendungen beispielsweise wird die Lebenszeit überlicherweise auf eine gewisse Jahreszahl begrenzt. 
Allerdings können wie auch bei den klassischen  Anwendungen bei einem Cloud-Anbieter Bedingungen eintreffen, die einen früheren Wechsel verlangen.

Diese Bedingungen können finanzieller Art sein. So könnte etwa der aktuelle Anbieter in Anbetracht seiner Leistungen nicht mehr wirtschaftlich sein.

Nicht nur finanzielle Aspekte können zu einem Wechselwunsch beim Kunden führen.

Durch Anpassungen am Leistungskatalog und die vertragliche Möglichkeit manche Leistungen nicht mehr anzubieten, kann es dazu kommen, dass notwenige Bausteine nicht mehr vom Cloud-Computing-Anbieter unterstützt werden.
Solche Anpassungen sind aufgrund der festen Vertragsregeln zwar nie plötzlich, meistens aber ein Argument für einen Wechsel.

Außerdem kann es dazu kommen, 
dass Kunden von mehreren Anbietern ihre benötigten Leistungen auf einen einzigen konsolidieren wollen oder im Gegenbeispiel ihre Anforderungen auf mehrere Anbieter verteilen wollen,
um die unterschiedlichen Alleinstellungsmerkmale mehrerer Anbieter gleichzeitig zu nutzen.

Zuletzt kann es auch durch äußere Faktoren wie gesetzliche Vorgaben,
denen das Produkt des aktuellen Anbieters nicht mehr folgt, dazu kommen,
dass ein Wechsel unbedingt notwendig wird.
Auch geopolitische Änderungen wie Zölle oder Gesetze zählen zu den Gründen für das frühzeitige Ende der Geschäftsbeziehung.

\paragraph{Strategische Bedeutung auf geopolitscher Ebene}

\subsection{Besonderheiten des Themas}

\paragraph{Unterschiede zu anderen Herausforderungen der Cloud}
Da darüberhinaus die Anbieterbindung nicht während der Lebenszeit einer Anbieterbeziehung sondern am Ende ins Gewicht fällt, ist die Auseinandersetztung nicht so allgegenwärtig wie andere Cloud-Themen.

\subsection{Beenden von Anbieterbeziehungen}

\section{Zielsetzung}

\subsection{Entwicklung einer Methode zur Analyse des Risikos von Anbieterbindung}
\paragraph{Verwendung von technischen Kriterien}
Vendor-Lockin ist ein primär technisches Problem für den Käufer eines Produktes be\-ziehungs\-weise konkret eins Cloud-Computing-Anbieters.

Daher wird untersucht welche technischen Kriterien zu diesem technischen Problem führen.
Technische Kriterien sind Eigenschaften eines Produktes im Kontext von Cloud-Computing, die sich auf die inherente Struktur und die Bestandteile des Produktes beziehen.


\section{Abgrenzungen}

\subsection{Distanzierung von ökonomischen Ansätzen}
Im Gegensatz dazu sind vertragliche oder ökonomische Kriterien Gegenstand dieser Arbeit.
Zur Verdeutlichung wird also beispielsweise nicht untersucht, ob die These, dass das Nutzen eines teurerer Cloud-Computing-Anbieter seltener zum Vendor-Lockin führt, zutrifft.

\subsection{Fokus auf öffentliche Cloud}

\subsection{Eingrenzung der zu analysierenden Anbieter}

\paragraph{Kennzahlen für Cloud-Anbieter}

\paragraph{Einblick in die Marktanteile in Deutschland}
Bei der Auswahl der Anbieter wurden sowohl solche berücksichtigt, die die kubus IT bereits verwendete, als auch solche die vermeintlich interessante Ergebnisse liefern sollten.
Aktuell sind folgende Cloud-Computing-Anbieter bereits in Benutzung.
\begin{itemize}
\item[-] Arvato
\item[-] Microsoft Azure
\end{itemize}
Darüberhinaus werden folgende Anbieter aufgrund ihrer Relevanz auf dem internationalen Markt, ihrer besonderen Größe oder ihrer Relevanz für deutsche Firmen berücksichtigt.
\begin{itemize}
\item[-] Amazon Warehoue Services (Vereinigte Staaten)
\item[-] Google Cloud Plattform (Vereinigte Staaten)
\item[-] Alibaba Cloud (China)
\item[-] IONOS Cloud (Deutschland)
\end{itemize}


