\thispagestyle{empty}
\section*{Kurzdarstellung}
\label{sec:kurzdarstellung}
Diese Arbeit ist in Zusammenarbeit zwischen dem IT-Dienstleister kubus IT und der Technischen Hochschule Georg Simon Ohm Nürnberg entstanden.

Ziel war es Anbieterbindung (englisch: Vendor-Lock-In) bei führenden Cloud-Computing-Anbietern insbesondere SaaS-Anbietern zu quantifizieren und zu vergleichen,
Dies wurde getan, um die Herausforderung der Anbieterbindung in den Auswahlprozess für Cloud-Computing-Anbieter integrieren zu können.

Dafür wurden einleitend die Begrifflichkeiten und deren Relevanz anhand von Fachliteratur und Umfragen unter Unternehmen erläutert.
Außerdem wurde eine Auswahl führender Cloud-Computing-Anbieter und deren Produktkataloge vorgestellt. 

Mit dieser Grundlage wurde die Motivation für die Beschäftigung mit Anbieterbindung und die Zielsetzung dieser Arbeit beschrieben.

Im Anschluss wurde der Kontext der gesetzlichen Krankenversicherung (GKV) in Deutschland eingeleitet und
die Ausgangssituation auch unter Berücksichtigung gesetzlicher Rahmenbedingungen und des aktuellen Vergabeverfahrens in der kubus IT dargestellt.

Danach wurde die vorschlagene Lösung zur Quantifizierung von Anbieterbindung, einem Bewertungsmodell, schrittweise entwickelt.

Zuletzt wurde das Bewertungsmodell auf eine Auswahl von Anbietern angewendet, die Ergebnisse aufgezeigt und deren Relevanz diskutiert.
Die Methodik der Arbeit wurde abschließend reflektiert,
eine Handlungsempfehlung für die kubus IT ausgesprochen 
und ein Ausblick auf die künftige Entwicklung der Thematik insbesondere im Bezug auf gesetzliche Änderungen gegeben.